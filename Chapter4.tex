% LaTeX source for textbook ``How to think like a computer scientist''
% Copyright (C) 1999  Allen B. Downey
% Copyright (C) 2009  Thomas Scheffler
% Copyright (C) 2023  Michael Penta

\chapter{Selection structures and recursion}

\label{condrecursion}


\section{Conditional expressions}
\index{control structures}
\index{conditional expression}
\index{condition}

In order to write useful programs, we almost always need the ability
to check certain conditions and change the behavior of the program
accordingly.  In C we can use {\bf control structures} to change the flow of our program.
Control structures utilize {\bf conditional expressions} to execute statements conditionally.

\index{operator!comparison}
\index{comparison!operator}

Conditional expressions are expressions that yield a true or false value. 
In C, any non-zero expression is true. Zero is false. The following are some example expressions and how they would evaluate in C

\begin{verbatim}
	'd'      /* true */
	1         /* true */
	-1       /* true */
	4.1     /* true */
	1 + 2  /* true*/
	0        /* false */
	-1 +1  /*false*/
	
\end{verbatim}
%

Most conditional expressions use {\bf comparison operators} 

\begin{verbatim}
	x == y               /* x equals y */
	x != y               /* x is not equal to y */
	x > y                /* x is greater than y */
	x < y                /* x is less than y */
	x >= y               /* x is greater than or equal to y */
	x <= y               /* x is less than or equal to y */
\end{verbatim}
%
Although these operations are probably familiar to you, the
syntax C uses is a little different from mathematical
symbols like $=$, $\neq$ and $\le$.  A common error is
to use a single {\tt =} instead of a double {\tt ==}.  Remember
that {\tt =} is the assignment operator, and {\tt ==} is
a comparison operator.  Also, there is no such thing as
{\tt =<} or {\tt =>}.

It is generally a good idea to make the two sides of a condition operator be the same
type so its best to compare {\tt int}s to {\tt int}s and {\tt double}s to {\tt double}s. With some implicit or explicit casting
you can also compare {\tt int}s with {\tt char}s and {\tt int}s with {\tt double}s of course you can always type cast if need.  
 Unfortunately, at this
point you can't compare strings at all!  There is
a way to compare strings, but we won't get to it for a couple
of chapters.

{\bf It is also important to note that you should only test floating point values using {\tt >} and {\tt <}.}

Due to the limitations of the floating point representation, these numbers cannot be compared for equality. 
If we need to test these for equality we have to determine if the numbers are close enough to each other. 
To calculate this, we first must decided on a tolerance (like .00001) and then compare  this to the difference of the two numbers.


\section{Selection structures: one-way} 
\index{statement!conditional}
\index{structures!selection}
\index{control structures!selection}
\index{one-way selcection}
\index{selection!one-way}

One category of control structures are the selection structures. The simplest selection structure is the {\tt if} statement:

\begin{verbatim}
  if (x > 0) 
  {
      printf ("x is positive\n");
  }
\end{verbatim}
%

The expression in parentheses must be a conditional expression.
If the expression is true, then the statements in brackets get executed.
If the condition is not true, the statements are not executed. 


This is considered one - way selection -- either you perform the task or you skip over it. 
There is only one choice and you select it or you do not.

\section{The modulus operator}
\index{modulus}
\index{operator!modulus}

The modulus operator works on integers (and integer expressions)
and yields the {\em remainder} when the first operand is divided
by the second.  In C, the modulus operator is a percent sign,
{\tt \%}.  The syntax is exactly the same as for other operators:

\begin{verbatim}
  int quotient = 7 / 3;
  int remainder = 7 % 3;
\end{verbatim}
%
The first operator, integer division, yields 2.  The second
operator yields 1.  Thus, 7 divided by 3 is 2 with 1 left over.

The modulus operator turns out to be surprisingly useful.  For
example, you can check whether one number is divisible by
another: if {\tt x \% y} is zero, then {\tt x} is divisible
by {\tt y}.

Also, you can use the modulus operator to extract the rightmost
digit or digits from a number.  For example, {\tt x \% 10} yields
the rightmost digit of {\tt x} (in base 10).  Similarly
{\tt x \% 100} yields the last two digits.


\section{Random numbers}
\label{Random numbers}
\label{random}
\label{pseudorandom}
\index{random number}
\index{deterministic}
\index{nondeterministic}
Most computer programs do the same thing every time they are executed,
so they are said to be {\bf deterministic}.  Usually, determinism is a
good thing, since we expect the same calculation to yield the same
result.  For some applications, though, we would like the
computer to be unpredictable.  Games are an obvious example.

Making a program truly {\bf nondeterministic} turns out to be not
so easy, but there are ways to make it at least seem
nondeterministic.  One of them is to generate {pseudorandom} numbers and
use them to determine the outcome of the program.
Pseudorandom numbers
are not truly random in the mathematical sense, but 
for our purposes, they will do.

\index{header file!stdlib.h}
\index{<stdlib.h>}

C provides a function called {\tt rand()} that generates
pseudorandom numbers.  It is declared in the
header file {\tt stdlib.h}, which contains a variety of ``standard
library'' functions, hence the name.

The return value from {\tt rand()} is an integer between 0 and {\tt
	RAND\_MAX}, where {\tt RAND\_MAX} is a large number (about 2 billion
on my computer) also defined in the header file.  Each time you call
{\tt rand()} you get a different randomly-generated number.  To see a
sample, run this:

\begin{verbatim}
 	int x = rand();
 	printf("%i\n", x);
 	x = rand();
 	printf("%i\n", x);
 	x = rand();
 	printf("%i\n", x);
 	x = rand();
 	printf("%i\n", x);
\end{verbatim}
%
On my machine I got the following output:

\begin{verbatim}
	1804289383
	846930886
	1681692777
	1714636915
\end{verbatim}
%
You will probably get something similar, but different, on yours.

Of course, we don't always want to work with gigantic integers.
More often we want to generate integers between 0 and some
upper bound.  A simple way to do that is with the modulus
operator.  For example:

\begin{verbatim}
 	int x = rand ();
 	int y = x % range + start
\end{verbatim}
%
Range here is the number of possible consecutive random values we would like. 
Start is the lowest random value we want.  Since {\tt y} is the remainder when {\tt x} is divided by
{\tt range}, the only possible values for {\tt y}
are between 0 and {\tt range - 1}, including both
end points.   Keep in mind, though, that {\tt y} will never
be equal to {\tt range}. The lower bound here is always 0 so if we add a start value we can shift the range to start at a new lower bound.

For example if we want a random number between 1 and 6, inclusively, our range is 6  [1, 2, 3, 4, 5, 6]. 
\begin{verbatim}
 	int range = 6;
 	int x = rand ();
 	int y = x % range;
\end{verbatim}
%
However, this code will generate a number between 0 and 5,  inclusively. To start this sequence at 1, we have to add a start value.

\begin{verbatim}
 	int range = 6;
 	int start = 1;
 	int x = rand ();
 	int y = x % range + start;
\end{verbatim}
%
It is also frequently useful to generate random floating-point values.
A common way to do that is by dividing by {\tt RAND\_MAX}.  For
example:

\begin{verbatim}
 	int x = rand ();
 	double y = (double) x / RAND_MAX;
\end{verbatim}
%
This code sets {\tt y} to a random value between 0.0 and 1.0,
including both end points.  As an exercise, you might want to
think about how to generate a random floating-point value in
a given range; for example, between 100.0 and 200.0.


\section{Random seeds}
\label{Random seeds}
\index{seed}
\index{random}

If you have run the code in this chapter a few times, you might
have noticed that you are getting the same ``random'' values
every time.  That's not very random!

One of the properties of pseudorandom number generators is that
if they start from the same place they will generate
the same sequence of values.  The starting place is called
a {\bf seed}; by default, C uses
the same seed every time you run the program.

While you are debugging, it is often helpful to
see the same sequence over and over.  That way, when you make
a change to the program you can compare the output before and
after the change.

If you want to choose a different seed for the random number
generator, you can use the {\tt srand()} function.  It takes
a single argument, which is an integer between 0 and {\tt RAND\_MAX}.

For many applications, like games, you want to see a different
random sequence every time the program runs.  A common way to
do that is to use a library function like {\tt time()}
to generate something reasonably unpredictable
and unrepeatable, like the number of seconds since January
1970, and use that number as a seed.  The details
of how to do that depend on your development environment but one example is shown here.

\begin{verbatim}
	#include <stdio.h>
	#include <stdlib.h>
	#include <time.h>
	
	int main (void)
	{
	 	// use the number of milliseconds from 1970 to seed rand
	 	 srand(time(0)); 
		
	 	//random number from 1 and 6 
	  	int x = rand() % 6 + 1;
	  	printf ("%i", x);
		
	  	//random number from -5 to positive 4
	  	x = rand() % 10 - 5;
	  	printf ("%i", x);
		
	  	return EXIT_SUCCESS;
	}
	
\end{verbatim}


Let's look at program that combines selection, modulus, and randomness. The following program generates a random number, either 1 for heads or 2 for tails. The result is printed.

\begin{verbatim}
 	#include <stdio.h>
 	#include <stdlib.h>
 	#include <time.h>
	
 	int main (void)
 	{
 		// use the number of milliseconds from 1970 to seed rand
	 	srand(time(0)); 
		
	 	//random number 1 2r 2
	 	int flip = rand() % 2 + 1;
 	
		if(flip == 1 )
		{
			puts("heads")
		}
		if(flip == 2 )
		{
	 	    puts("tails")
		}
		
	 	return EXIT_SUCCESS;
 	}
	
\end{verbatim}



\section {Selection structures: two-way}
\label{alternative}
\index{conditional!alternative}
\index{structures!selection}
\index{control structures!selection}
\index{two-way selcection}
\index{selection!two-way}

A second form of the selection structures is two-way selection, 
in which there are two possibilities, and the condition determines
which one gets executed.  We do one thing or we do another thing, 
unlike one-way selection where we did the thing or we didn't do the thing
 The syntax looks like:

\begin{verbatim}
  if (x%2 == 0)
  {
      printf ("x is even\n");
  } 
  else 
  {
      printf ("x is odd\n");
  }
\end{verbatim}
%
If the remainder when {\tt x} is divided by 2 is zero, then
we know that {\tt x} is even, and this code displays a message
to that effect.  If the condition is false, the second
set of statements is executed.  Since the condition must
be true or false, exactly one of the alternatives will be
executed.

As an aside, if you think you might want to check the parity
(evenness or oddness) of numbers often, you might want to
``wrap'' this code up in a function, as follows:

\begin{verbatim}
  void printParity (int x) 
  {
      if (x%2 == 0) 
      {
          printf ("x is even\n");
      } 
      else 
      {
          printf ("x is odd\n");
      }
  }
\end{verbatim}
%
Now you have a function named {\tt printParity()} that will display
an appropriate message for any integer you care to provide.
In {\tt main()} you would call this function as follows:

\begin{verbatim}
   printParity (17);
\end{verbatim}
%
Always remember that when you {\em call} a function, you do
not have to declare the types of the arguments you provide.
C can figure out what type they are based on the definition and the prototype.
You should resist the temptation to write things like:

\begin{verbatim}
   int number = 17;
   printParity (int number);         /* WRONG!!! */
\end{verbatim}

\section {Chaining}
\index{chain}
\index{selection!chaining}

Sometimes you want to check for a number of related conditions
and choose one of several actions.  One way to do this is by
{\bf chaining} a series of {\tt if}s and {\tt else}s:

\begin{verbatim}
    if (x > 0) 
    {
        printf ("x is positive\n");
    } 
    else if (x < 0) 
    {
        printf ("x is negative\n");
    } 
    else 
    {
        printf ("x is zero\n");
    }
\end{verbatim}
%
These chains can be as long as you want, although they can
be difficult to read if they get out of hand.  One way to
make them easier to read is to use standard indentation,
as demonstrated in these examples.  If you keep all the
statements and squiggly-braces lined up, you are less
likely to make syntax errors and you can find them more
quickly if you do.

\section{Nested conditionals}
\index{conditional!nested}

In addition to chaining, you can also nest one control structures
within another.  We could have written the previous example
as:

\begin{verbatim}
    if (x == 0) 
    {
        printf ("x is zero\n");
    } 
    else 
    {
        if (x > 0) 
        {
            printf ("x is positive\n");
        }
        else 
        {
            printf ("x is negative\n");
        }
    }
\end{verbatim}
%
There is now an outer conditional that contains two branches.  The
first branch contains a simple output statement, but the second
branch contains another {\tt if} statement, which has two branches
of its own.  Fortunately, those two branches are both output
statements, although they could have been conditional statements as
well.

Notice again that indentation helps make the structure
apparent, but nevertheless, nested conditionals get difficult to read
very quickly.  In general, it is a good idea to avoid them when you
can.

\index{nested structure}

On the other hand, this kind of {\bf nested structure} is common, and
we will see it again, so you better get used to it.


\section{Selection structures: switch}
\index{switch}
\index{selection!switch}
\index{control structure!switch}

A switch is sort of short hand to replace some long chained structures. 
You can use a switch when you are testing a single {\tt int} or {\tt char}, you
are testing for equality, and you would otherwise have a long chain. For example we can take this chained structure that is testing the 
int value and printing a message based on the value.

\begin{verbatim}
	
	//x is an int
	
	if (x == 1) 
	{
		puts("message1");
	} 
	else if (x == 2) 
	{
		puts("message2");
	} 
	else if (x == 3) 
	{
		puts("message3");
	} 
	else 
	{
		puts("default message");
	}
\end{verbatim}
%

We can rewrite this as a switch because we are testing an int value for equality in a chained structure. Each block of code in the chained structure 
becomes a case in the switch. Each case in the switch must have a label and a break. Most labels start with {\tt case|} and then have the value we are testing for equality --
like {\tt case 2:} or {\tt case 'c':}. One case has a unique label -- {\tt default}. The default case has no value to test because it is the catchall -- it only runs if all other cases fail. All cases end with the word {\tt break}. A break forces control to leave the switch structure. Breaks are an important part of the switch structure.
Write a program that prompts the user for an integer. Use the switch to print various messages. Try removing some of the breaks and  check out how the changes behave.
Omitting breaks between cases can cause fall through - this can be a bug or a feature depending on how you use it (look into stacking cases in switches).


\begin{verbatim}
	
	//x is an int
	
 switch(x)
 {
  case 1:
   puts("message1");
   break;

  case 2:
   puts("message2");
  break;

  case 3:
   puts("message3");
   break;

 default:
  puts("message4");
  break;
		
	\end{verbatim}
	%

\section{The {\tt return} statement and early termination}
\index{return}
\index{statement!return}

The {\tt return} statement allows you to terminate the execution
of a function. You can place a return statement in any part of the function. Once the program hits the return it will leave the function and go back to the caller. 
If you put a return statement before you before you reach the end of a function this is called an early return or early termination. This can be useful to guard the function from doing unnecessary action. For example, if the parameter of the function must greater than zero, then we can put an if statement to act as a gaurd and stop the 
execution of the function right off the top.

\begin{verbatim}
  #include <math.h>

  void printLogarithm (double x) 
  {
      if (x > 0.0) 			 /*'guard' for early return if x >0 */
      {
          printf ("Positive numbers only, please.\n");
          return;  
      }

     double result = log (x);
     printf ("The log of x is %f\n", result);
  }
\end{verbatim}
%
This defines a function named {\tt printLogarithm()} that takes
a {\tt double} named {\tt x} as a parameter.  The first thing
it does is check whether {\tt x} is greater than
zero, in which case it displays an error message and then uses
{\tt return} to exit the function.  The flow of execution
immediately returns to the caller and the remaining lines of
the function are not executed.

Remember that any time you want to use one a function from the math
library, you have to include the header file {\tt math.h} and 
you may be required to link the math library at compile time with -lm (dash L M)

\section{Recursion}
\label{recursion}
\index{recursion}

I mentioned in the last chapter that it is legal for one function to
call another, and we have seen several examples of that.  I neglected
to mention that it is also legal for a function to call itself.  It
may not be obvious why that is a good thing, but it turns out to be
one of the most magical and interesting things a program can do.

For example, look at the following function:

\begin{verbatim}
  void countdown (int n) 
  {
      if (n == 0) 
      {
          printf ("Blastoff!");
      }
      else
      {
          printf ("%i", n);
          countdown (n-1);
      }
  }
\end{verbatim}
%
The name of the function is {\tt countdown()} and it takes a single
integer as a parameter.  If the parameter is zero, it outputs
the word ``Blastoff.''  Otherwise, it outputs the parameter and
then calls a function named {\tt countdown()}---itself---passing
{\tt n-1} as an argument.

What happens if we call this function like this:

\begin{verbatim}
  int main (void)
  {
       countdown (3);
       return EXIT_SUCCESS;
  }
\end{verbatim}
%
The execution of {\tt countdown()} begins with {\tt n=3}, and
since {\tt n} is not zero, it outputs the value 3, and then
calls itself...

\begin{quote}
The execution of {\tt countdown()} begins with {\tt n=2}, and
since {\tt n} is not zero, it outputs the value 2, and then
calls itself...

\begin{quote}
The execution of {\tt countdown()} begins with {\tt n=1}, and
since {\tt n} is not zero, it outputs the value 1, and then
calls itself...

\begin{quote}
The execution of {\tt countdown()} begins with {\tt n=0}, and
since {\tt n} is zero, it outputs the word ``Blastoff!''
and then returns.
\end{quote}

The countdown that got {\tt n=1} returns.

\end{quote}

The countdown that got {\tt n=2} returns.

\end{quote}

The countdown that got {\tt n=3} returns.

\noindent And then you're back in {\tt main()} (what a trip).  So the
total output looks like:

\begin{verbatim}
    3
    2
    1
    Blastoff!
\end{verbatim}
%
As a second example, let's look again at the functions
{\tt printNewLine()} and {\tt printThreeLines()}.

\begin{verbatim}
  void printNewLine () 
    {
        printf ("\n");
    }

  void printThreeLines () 
    {
        printNewLine ();  printNewLine ();  printNewLine ();
    }
\end{verbatim}
%
Although these work, they would not be much help if I wanted
to output 2 newlines, or 106.  A better alternative would be

\begin{verbatim}
    void printLines (int n) 
    {
        if (n > 0) 
        {
            printf ("\n");
            printLines (n-1);
        }
    }
\end{verbatim}
%
This program is similar to {\tt countdown}; as long as {\tt n} is
greater than zero, it outputs one newline, and then calls itself to
output {\tt n-1} additional newlines.  Thus, the total number of
newlines is {\tt 1 + (n-1)}, which usually comes out to roughly {\tt
n}.

\index{recursive}
\index{newline}

The process of a function calling itself is called {\bf recursion}, and
such functions are said to be {\bf recursive}.

\section {Infinite recursion}

In the examples in the previous section, notice that each time the
functions get called recursively, the argument gets smaller by one, so
eventually it gets to zero.  When the argument is zero, the function
returns immediately, {\em without making any recursive calls}.
This case---when the function completes without making a recursive
call---is called the {\bf base case}.

If a recursion never reaches a base case, it will go on making recursive
calls forever and the program will never terminate.  This is known as
{\bf infinite recursion}, and it is generally not considered a good
idea.

\index{recursion!infinite}
\index{infinite recursion}
\index{run-time error}

In most programming environments, a program with an infinite
recursion will not really run forever.  Eventually, something
will break and the program will report an error.  This is the
first example we have seen of a run-time error (an error that
does not appear until you run the program).

You should write a small program that recurses forever and run
it to see what happens.

\section {Tips on writing recursion solutions}
It is helpful to have a strategy to tackle recursive solutions. 
\begin{enumerate}
	\item Identify a base case. This is what stops the recursion. This block will not have a recursive call
	\item Identify the general case (the recursive step). This is the repeating part. It will contain a recursive call
	\item The recursive call will use an argument that gets the next step closer to the base case.
	\item An if statement will test the parameter and determine if the base case is run or the general case.
\end{enumerate}

\section {Stack diagrams for recursive functions}
\index{stack}
\index{stack diagram}
\index{diagram!state}
\index{diagram!stack}

In the previous chapter we used a stack diagram to represent the
state of a program during a function call.  The same kind
of diagram can make it easier to interpret a recursive function.

Remember that every time a function gets called it creates
a new instance that contains
the function's local variables and parameters.

This figure shows a stack diagram for Countdown, called
with {\tt n = 3}:

\vspace{0.1in}
\centerline{\epsfig{figure=figs/stack2.pdf,width=6cm}}
\vspace{0.1in}
%
There is one instance of {\tt main()} and four instances of
{\tt Countdown()}, each with a different value for the parameter
{\tt n}.  The bottom of the stack, {\tt Countdown()} with {\tt n=0}
is the base case.  It does not make a recursive call, so there
are no more instances of {\tt Countdown()}.

The instance of {\tt main()} is empty because {\tt main()} does not
have any parameters or local variables.  As an exercise, draw a
stack diagram for {\tt PrintLines()}, invoked with the parameter {\tt n=4}.


\section{Glossary}

\begin{description}

\item[modulus:]  An operator that works on integers and yields
the remainder when one number is divided by another.  In C
it is denoted with a percent sign ({\tt \%}).

\item[deterministic:]  A program that does the same thing every
time it is run.

\item[pseudorandom:]  A sequence of numbers that appear to be
random, but which are actually the product of a deterministic
computation.

\item[seed:]  A value used to initialize a random number sequence.
Using the same seed should yield the same sequence of values.

\item[condition:]  An expression that results in true or false


\item[control structure:]  A structure that uses a condition to allow for conditionally executing a block of code. 

\item[selection structure:]  A type of control structure. Selection structures include {\tt if}, {\tt if...else}, and {\tt switch}

\item[chaining:]  A way of joining several conditional statements
in sequence.

\item[nesting:] Putting a conditional statement inside one or both
branches of another conditional statement.

\item[recursion:]  The process of calling the same function you
are currently executing.

\item[infinite recursion:]  A function that calls itself
recursively without every reaching the base case.  Eventually
an infinite recursion will cause a run-time error.

\index{modulus}
\index{conditional}

\index{conditional!chained}
\index{conditional!nested}
\index{recursion}
\index{recursion!infinite}
\index{infinite recursion}
\index{random}
\index{random!seed}
\end{description}


\section{Exercises}
\setcounter{exercisenum}{0}


% LaTeX source for textbook ``How to think like a computer scientist''
% Copyright (C) 1999  Allen B. Downey
% Copyright (C) 2009  Thomas Scheffler

\begin{exercise}
Use the variables initialized below to evaluate each expression. State if C would consider the resulting value as true or false.
\begin{verbatim}
	int m = 0;
	int n = 1;
	char p = 'k';
	char s = 'F'
	double t = 1.2    
\end{verbatim}
\begin{enumerate}
	\item {\tt m / n}
	\item {\tt m + n}
	\item {\tt m}
	\item {\tt p <= s}
	\item {\tt t < n}
\end{enumerate}
\end{exercise}

\begin{exercise}
This exercise reviews the flow of execution through a program
with multiple methods.  Read the following code and answer the
questions below.

\begin{verbatim}

   #include <stdio.h>
   #include <stdlib.h>
  
   void zippo (int, int);
   void baffle (int);
  
   int main (void)
   {
  	 zippo (5, 13);
  	 return EXIT_SUCCESS;
   }
   void baffle (int output)
   {
  	 printf ("%i\n",output);
  	 zippo (12, -5);
   }
  
   void zippo (int quince, int flag)
   {
    	if (flag < 0)
    	{
  	    	printf ("%i zoop\n", quince);
     	}
    	else
    	{
  	    	printf ("rattle ");
  	    	baffle (quince);
  	    	printf ("boo-wa-ha-ha\n");
    	}
    }
\end{verbatim}
%
\begin{enumerate}

\item Write the number {\tt 1} next to the first {\em statement}
of this program that will be executed.  Be careful to distinguish
things that are statements from things that are not.

\item Write the number {\tt 2} next to the second statement, and so on
until the end of the program.  If a statement is executed more than
once, it might end up with more than one number next to it.

\item What is the value of the parameter {\tt quince} when {\tt baffle()}
gets invoked for the first time?

\item What is the exact output of this program? Pay close attention to the printed white space like spaces, tabs, and new lines.

\end{enumerate}

\end{exercise}

\begin{exercise}
	In this exercise you will practice using random with functions
	
		\begin{enumerate}
		\item Define a function and a prototype called rollDie that has no parameters. 
	In the function call rand to generate a random number 1, 2, 3, 4, 5, or 6. Print the resulting value.
	
	\item Define a main function that seeds the random with epoch time (use time() from time.h) and calls your function 3 times
		\end{enumerate}
\end{exercise}

\begin{exercise}
	In this exercise you will practice using selection statements with functions
	\begin{enumerate}
		\item Define a function and a prototype called validate that has one parameter, an int. 
		If the int is 0 the function should print an error message and terminate. 
		If the parameter is non-zero, multiply the value by 2 and print the result.
		
		\item 
		Define a main function that calls your function 3 times with the following arguments: 0, 3, -2
		
	\end{enumerate}

\end{exercise}


\begin{exercise}
There is an old song about 
beer bottles that can be expressed recursively.

The first verse of the song ``99 Bottles of Beer'' is:

\begin{quote}
99 bottles of beer on the wall,
99 bottles of beer,
ya' take one down, ya' pass it around,
98 bottles of beer on the wall.
\end{quote}

Subsequent verses are identical except that the number
of bottles gets smaller by one in each verse, until the
last verse:

\begin{quote}
No bottles of beer on the wall,
no bottles of beer,
ya' can't take one down, ya' can't pass it around,
'cause there are no more bottles of beer on the wall!
\end{quote}
%
And then the song (finally) ends.

Write a program that prints the entire lyrics of
``99 Bottles of Beer.''  Your program should include a
recursive method that does the hard part, but you also
might want to write additional methods to separate the major
functions of the program.

The last verse, when the number of bottles left is 0, is the base case. 
The other verses are the recursive step. 

As you are developing your code, you will probably
want to test it with a small number of verses, like
``3 Bottles of Beer.''

The purpose of this exercise is to take a problem and break it
into smaller problems, and to solve the smaller problems by writing
simple, easily-debugged methods.
\end{exercise}


\begin{exercise}
You can use the  {\tt getchar()} function in C to
get character input from the user through the keyboard.
This function stops the execution of the program and waits
for the input from the user. 

The {\tt getchar()} function has the type {\tt int} and does
not require an argument. It returns the ASCII-Code (cf. Appendix~\ref{ASCII-Table})
of the key that has been pressed on the keyboard.
\begin{enumerate}
\item Write a program, that asks the user 
to input a digit between  0 and 9. 

\item Test the input from the user and display an error message 
if the returned value is not a digit. The program should then
be terminated.
If the test is successful, the program should print the 
input value on the computer screen.
\end{enumerate}


\end{exercise}

% Kapitel 5 (Return)
%Schreiben Sie dazu eine Funktion {\tt AsciiToNumber()} welche ein
%{\tt int} als Typ und als Argument besitzt. �bergeben Sie der Funktion
%den eingelesenen Wert und 




% String--Ausgabe!

%\begin{exercise}
%Lesen Sie das folgende Programm. Notieren Sie die Ausgabe die w�hrend der
%Abarbeitung des Programms erzeugt wird.

%\begin{verbatim}
%  void Zoop (String fred, int bob) 
%  {
%        printf ("%s\n", fred);
%        if (bob == 5) 
%        {
%            ping ("not ");
%        } 
%        else 
%        {
%            printf ("!\n");
%        }
%    }

%  int main (void) 
%  {
%        int bizz = 5;
%        int buzz = 2;
%        zoop ("just for", bizz);
%        clink (2*buzz);
%    }

%  void clink (int fork) 
%  {
%        printf ("It's ");
%        zoop ("breakfast ", fork) ;
%  }

%  void ping (String strangStrung) 
%  {
%        printf ("any %smore \n", strangStrung);
%    }
%}
%\end{verbatim}
%\end{exercise}





