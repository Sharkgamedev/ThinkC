% LaTeX source for textbook ``How to think like a computer scientist''
% Copyright (C) 1999  Allen B. Downey
% Copyright (C) 2009  Thomas Scheffler

\setcounter{chapter}{2}
\chapter{Function}

\section{Floating-point}
\index{floating-point number}
\index{type!double}
\index{double (floating-point)}

In the last chapter we had some problems dealing with numbers
that were not integers.  We worked around the problem by measuring
percentages instead of fractions, but a more general solution is
to use floating-point numbers, which can represent fractions
as well as integers.  In C, there are two floating-point types,
called {\tt float} and {\tt double}.  In this book we will use
{\tt double}s exclusively.

You can create floating-point variables and assign values to them
using the same syntax we used for the other types.  For example:

\begin{verbatim}
  double pi;
  pi = 3.14159;
\end{verbatim}
%
It is also legal to declare a variable and assign a value to it at the
same time:

\begin{verbatim}
  int x = 1;
  char first_char = "a";
  double pi = 3.14159;
\end{verbatim}
%
In fact, this syntax is quite common.  A combined declaration
and assignment is sometimes called an {\bf initialization}.

\index{initialization}

Although floating-point numbers are useful, they are
often a source of confusion because there seems to be an
overlap between integers and floating-point numbers.  For
example, if you have the value {\tt 1}, is that an integer,
a floating-point number, or both?

Strictly speaking, C distinguishes the integer value {\tt 1}
from the floating-point value {\tt 1.0}, even though they
seem to be the same number.  They belong to
different types, and strictly speaking, you are not allowed
to make assignments between types.  For example, the following
is illegal:

\begin{verbatim}
    int x = 1.1;
\end{verbatim}
%
Because the variable on the left is an {\tt int}
and the value on the right is a {\tt double}.  But it is easy
to forget this rule, especially because there are places where C
automatically converts from one type to another.
For example,

\begin{verbatim}
    double y = 1;
\end{verbatim}
%
should technically not be legal, but C allows it by converting the
{\tt int} to a {\tt double} automatically.  This is convenient for the programmer,
but it can cause problems; for example:

\begin{verbatim}
    double y = 1 / 3;
\end{verbatim}
%
You might expect the variable {\tt y} to be given the value
{\tt 0.333333}, which is a legal floating-point value, but in
fact it will get the value {\tt 0.0}.  The reason is that the
expression on the right appears to be the ratio of two integers,
so C does {\em integer} division, which yields the integer
value {\tt 0}.  Converted to floating-point, the result is
{\tt 0.0}.

One way to solve this problem (once you figure out what
it is) is to make the right-hand side a floating-point
expression:

\begin{verbatim}
    double y = 1.0 / 3.0;
\end{verbatim}
%
This sets {\tt y} to {\tt 0.333333}, as expected.

\index{arithmetic!floating-point}

All the operations we have seen---addition, subtraction,
multiplication, and division---work on floating-point values,
although you might be interested to know that the underlying mechanism
is completely different.  In fact, most processors have special
hardware just for performing floating-point operations.

\section{Constants}
\label{sec:Constants}
\index{constants}
\index{constant values}

In the previous section we have assigned the value 3.14159 to a
floating point variable. An important thing to remember about variables
is, that they can hold -- as their name implies -- 
different values at different points in your program. 
For example, we could assign
the value 3.14159 to the variable {\tt pi} now and assign 
some other value to it later on:

\begin{verbatim}
    double pi = 3.14159;
    ...
    pi = 10.999;  /* probably a logical error in your program */
\end{verbatim}
%

The second value is probably not what you intended when you first created the named
storage location {\tt pi} . The value for $\pi$ is  constant and does not
change over time. Using the storage location {\tt pi} to hold arbitrary other values can 
cause some very hard to find bugs in your program.

C allows you to specify the static nature of storage locations through
the use of the keyword {\tt const}. It must be used in conjunction with the 
required type of the constant. A value will be assigned at initialisation but can
never be changed again during the runtime of the program.  

\begin{verbatim}
    const double PI = 3.14159;
    printf ("Pi: %f\n", PI);
     ...
    PI = 10.999;  /* wrong, error caught by the compiler  */
\end{verbatim}
%

It is no longer possible to change the value for  {\tt PI} once it has been initialised, but 
other than this we can use it just like a variable.

In order to visually separate constants from variables we will use
all uppercase letters in their names. 



%todo: typecasting �berarbeiten  - siehe deutsche Folien

\section{Converting from {\tt double} to {\tt int}}
\label{rounding}
\label{typecasting} 
\index{rounding}
\index{typecasting}

As I mentioned, C converts {\tt int}s
to {\tt double}s automatically if necessary, because no
information is lost in the translation.  On the other hand,
going from a {\tt double} to an {\tt int} requires rounding
off.  C doesn't perform this operation automatically, in
order to make sure that you, as the programmer, are aware
of the loss of the fractional part of the number.

The simplest way to convert a floating-point value to an integer is to
use a {\bf typecast}.  Typecasting is so called because it allows you
to take a value that belongs to one type and ``cast'' it into another
type (in the sense of molding or reforming, not throwing).

The syntax for typecasting requires the explicit specification of
the target type, set in parenthesis before the expression {\tt (Type)}.
For example:

\begin{verbatim}
  const double PI = 3.14159;
  int x = (int) PI;
\end{verbatim}
%
The {\tt (int)} operator casts the value of PI into an integer, so {\tt x} gets the value
3.  Converting to an integer always rounds down, even if the fraction
part is 0.99999999.

For every type in C, there is a corresponding operator that
typecasts its argument to the appropriate type.

\section{Math functions}
\index{math function}
\index{function!math}
\index{expression}
\index{argument}

In mathematics, you have probably seen functions like $\sin$ and
$\log$, and you have learned to evaluate expressions like
$\sin(\pi/2)$ and $\log(1/x)$.  First, you evaluate the
expression in parentheses, which is called the {\bf argument} of the
function.  For example, $\pi/2$ is approximately 1.571, and $1/x$ is
0.1 (if $x$ happens to be 10).

Then you can evaluate the function itself, either by looking it up in
a table or by performing various computations.  The $\sin$ of 1.571 is
1, and the $\log$ of 0.1 is -1 (assuming that $\log$ indicates the
logarithm base 10).

This process can be applied repeatedly to evaluate more complicated
expressions like $\log(1/\sin(\pi/2))$.  First we evaluate the
argument of the innermost function, then evaluate the function,
and so on.

C provides a set of built-in functions that includes most
of the mathematical operations you can think of.
The math functions are invoked using a syntax that is similar to
mathematical notation:

\begin{verbatim}
     double log = log (17.0);
     double angle = 1.5;
     double height = sin (angle);
\end{verbatim}
%
The first example sets {\tt log} to the logarithm of 17, base
$e$.  There is also a function called {\tt log10} that takes
logarithms base 10.

The second example finds the sine of the value of the variable
{\tt angle}.  C assumes that the
values you use with {\tt sin} and the other trigonometric functions
({\tt cos}, {\tt tan}) are in {\em radians}.  To
convert from degrees to radians, you can divide by 360
and multiply by $2 \pi$.  

If you don't happen to know $\pi$ to 15 digits, you can
calculate it using the {\tt acos} function.  The arccosine
(or inverse cosine) of -1 is $\pi$, because the cosine of
$\pi$ is -1.

\begin{verbatim}
  const double PI = acos(-1.0);
  double degrees = 90;
  double angle = degrees * 2 * PI / 360.0;
\end{verbatim}
%
Before you can use any of the math functions, you have to
include the math {\bf header file}.  Header files contain
information the compiler needs about functions that are defined
outside your program.  For example, in the ``Hello, world!''
program we included a header file named {\tt stdio.h} using
an {\bf include} statement:

\begin{verbatim}
#include <stdio.h>
\end{verbatim}
%
{\tt stdio.h} contains information about input and output
(I/O) functions available in C.

\index{header file}
\index{<stdio.h>}
\index{<math.h>}
\index{header file!stdio.h}
\index{header file!math.h}

Similarly, the math header file contains information
about the math functions.  You can include it at the beginning
of your program along with {\tt stdio.h}:

\begin{verbatim}
#include <math.h>
\end{verbatim}

\section {Composition}
\label{composition}
\index{composition}
\index{expression}

Just as with mathematical functions, C functions can be {\bf
composed}, meaning that you use one expression as part of another.
For example, you can use any expression as an argument to a function:

\begin{verbatim}
    double x = cos (angle + PI/2);
\end{verbatim}
%
This statement takes the value of {\tt PI}, divides it by two and
adds the result to the value of {\tt angle}.  The sum is
then passed as an argument to the {\tt cos} function.

You can also take the result of one function and pass it as
an argument to another:

\begin{verbatim}
    double x = exp (log (10.0));
\end{verbatim}
%
This
statement finds the log base $e$ of 10 and then raises $e$ to that
power.  The result gets assigned to {\tt x}; I hope you know what it
is.

\section{Adding new functions}
\index{function!definition}
\index{main}
\index{function!main}

So far we have only been using the functions that are built into C,
but it is also possible to add new functions.  Actually, we have already
seen one function definition: {\tt main()}.  The function named {\tt main()}
is special because it indicates where the execution of the program
begins, but the syntax for {\tt main()} is the same as for any other
function definition:

\begin{verbatim}
  void NAME ( LIST OF PARAMETERS ) 
  {
      STATEMENTS
  }
\end{verbatim}
%
You can make up any name you want for your function, except
that you can't call it {\tt main} or any other
C keyword.  The list of
parameters specifies what information, if any, you have to
provide in order to use (or {\bf call}) the new function.

{\tt main()} doesn't take any parameters, as indicated by
the parentheses containing the keyword {\tt (void)} in it's definition.  The first couple
of functions we are going to write also have no parameters, so the
syntax looks like this:

\begin{verbatim}
  void PrintNewLine (void) 
  {
      printf ("\n");
  }
\end{verbatim}
%
This function is named {\tt PrintNewLine()}. It contains only a single
statement, which outputs a newline character. Notice that we start
the function name with an uppercase letter. The following words of the
function name are also capitalized. We will use this convention for the naming 
of functions consistently throughout the book.

In {\tt main()} we can call this new function using syntax that
is similar to the way we call the built-in C commands:

\begin{verbatim}
  int main (void)
  {
     printf ("First Line.\n");
     PrintNewLine ();
     printf ("Second Line.\n");
     return EXIT_SUCCESS;
  }
\end{verbatim}
%
The output of this program is:
\vskip 0.5em

\begin{verbatim}
   First line.

   Second line.
\end{verbatim}
%
Notice the extra space between the two lines.  What if we wanted
more space between the lines?  We could call the same
function repeatedly:

\begin{verbatim}
  int main (void)
  {
      printf ("First Line.\n");
      NewLine ();
      NewLine ();
      NewLine ();
      printf ("Second Line.\n");
  }
\end{verbatim}
%
Or we could write a new function, named {\tt PrintThreeLines()}, that 
prints three new lines:

\begin{verbatim}
  void PrintThreeLines (void)
  {
    PrintNewLine ();  PrintNewLine ();  PrintNewLine ();
  }

  int main (void)
  {
      printf ("First Line.\n");
      PrintThreeLines ();
      printf ("Second Line.\n");
      return EXIT_SUCCESS;
  }
\end{verbatim}
%
You should notice a few things about this program:

\begin{itemize}

\item You can call the same procedure repeatedly.  In
fact, it is quite common and useful to do so.

\item You can have one function call another function.  In this
case, {\tt main()} calls {\tt PrintThreeLines()} and {\tt PrintThreeLines()}
calls {\tt PrintNewLine()}.  Again, this is common and useful.

\item In {\tt PrintThreeLines()} I wrote three statements all on the
same line, which is syntactically legal (remember that spaces
and new lines usually don't change the meaning of a program).
On the other hand, it is usually a better idea to put each
statement on a line by itself, to make your program easy to
read.  I sometimes break that rule in this book to save space.

\end{itemize}

So far, it may not be clear why it is worth the trouble to
create all these new functions.  Actually, there are a lot
of reasons, but this example only demonstrates two:

\begin{enumerate}

\item Creating a new function gives you an opportunity to
give a name to a group of statements.  Functions can simplify a program
by hiding a complex computation behind a single command, and by using
English words in place of arcane code.  Which is clearer, {\tt
PrintNewLine()} or {\tt printf("$\backslash$n")}?

\item Creating a new function can make a program smaller by eliminating
repetitive code.  For example, a short way to print nine consecutive
new lines is to call {\tt PrintThreeLines()} three times.  How would you
print 27 new lines?

\end{enumerate}

\section {Definitions and uses}

Pulling together all the code fragments from the previous
section, the whole program looks like this:

\begin{verbatim}
  #include <stdio.h>
  #include <stdlib.h>

  void PrintNewLine (void) 
  {
      printf ("\n");
  }

  void PrintThreeLines (void)
  {
      PrintNewLine ();  PrintNewLine ();  PrintNewLine ();
  }

  int main (void)
  {
     printf ("First Line.\n");
     PrintThreeLines ();
     printf ("Second Line.\n");
     return EXIT_SUCCESS;
  }

\end{verbatim}

This program contains three function definitions: {\tt PrintNewLine()},
{\tt PrintThreeLine()}, and {\tt main()}.

Inside the definition of {\tt main()}, there is a statement that
uses or calls {\tt PrintThreeLine()}.  Similarly, {\tt PrintThreeLine()} calls
{\tt PrintNewLine()} three times.  Notice that the definition of each
function appears above the place where it is used.

This is necessary in C; the definition of a function must
appear before (above) the first use of the function.  You
should try compiling this program with the functions in a
different order and see what error messages you get.

\section {Programs with multiple functions}

When you look at the C source code that contains several functions, it
is tempting to read it from top to bottom, but that is likely to be
confusing, because that is not the {\bf order of execution} of the
program.

Execution always begins at the first statement of {\tt main()},
regardless of where it is in the program (often it is at the bottom).
Statements are executed one at a time, in order, until you reach a
function call.  Function calls are like a detour in the flow of
execution.  Instead of going to the next statement, you go to the
first line of the called function, execute all the statements there,
and then come back and pick up again where you left off.

That sounds simple enough, except that you have to remember that one
function can call another.  Thus, while we are in the middle of {\tt
main()}, we might have to go off and execute the statements in {\tt
PrintThreeLines()}.  But while we are executing {\tt PrintThreeLines()}, we get
interrupted three times to go off and execute {\tt PrintNewLine()}.

Fortunately, C is adept at keeping track of where it is, so
each time {\tt PrintNewLine()} completes, the program picks up where it left
off in {\tt PrintThreeLine()}, and eventually gets back to {\tt main()} so the
program can terminate.

What's the moral of this sordid tale?  When you read a program, don't
read from top to bottom.  Instead, follow the flow of execution.

\section {Parameters and arguments}
\index{parameter}
\index{argument}

Some of the built-in functions we have used have {\bf parameters},
which are values that you provide to let the function do its
job.  For example, if you want to find the sine of a number,
you have to indicate what the number is.  Thus, {\tt sin()}
takes a {\tt double} value as a parameter.

Some functions take more than one parameter, like {\tt pow()},
which takes two {\tt doubles}, the base and the exponent.

Notice that in each of these cases we have to specify not
only how many parameters there are, but also what type they
are.  So it shouldn't surprise you that when you write a
function definition, the parameter list indicates the type of
each parameter.  For example:

\begin{verbatim}
  void PrintTwice (char phil) 
  {
      printf("%c%c\n", phil, phil);
  }
\end{verbatim}
%
This function takes a single parameter, named {\tt phil}, that
has type {\tt char}.  Whatever that parameter is (and at
this point we have no idea what it is), it gets printed
twice, followed by a newline.
I chose the name {\tt phil} to suggest that the name
you give a parameter is up to you, but in general you want to
choose something more illustrative than {\tt phil}.

In order to call this function, we have to provide a {\tt char}.
For example, we might have a {\tt main()} function like this:

\begin{verbatim}
  int main (void) 
  {
      PrintTwice ('a');
      return EXIT_SUCCESS;
  }
\end{verbatim}
%
The {\tt char} value you provide is called an {\bf argument}, and we
say that the argument is {\bf passed} to the function.  In this
case the value {\tt 'a'} is passed as an argument
to {\tt PrintTwice()} where it will get printed twice.

Alternatively, if we had a {\tt char} variable, we could
use it as an argument instead:

\begin{verbatim}
  int main () 
  {
      char argument = 'b';
      PrintTwice (argument);
      return EXIT_SUCCESS;
  }
\end{verbatim}
%
Notice something very important here: the name of the variable we pass
as an argument ({\tt argument}) has nothing to do with the name of the
parameter ({\tt phil}).  Let me say that again:

\begin{quote}

{\bf The name of the variable we pass as an argument has nothing to do
with the name of the parameter.}

\end{quote}

They can be the same or they can be different, but it is important
to realize that they are not the same thing, except that they happen
to have the same value (in this case the character {\tt 'b'}).

The value you provide as an argument must have the same type as
the parameter of the function you call.  This rule is
important, but it is sometimes confusing because C sometimes
converts arguments from one type to another automatically.  For
now you should learn the general rule, and we will deal with
exceptions later.

\section {Parameters and variables are local}

Parameters and
variables only exist inside their own functions.  Within the
confines of {\tt main()}, there is no such thing as {\tt phil}.
If you try to use it, the compiler will complain.  Similarly,
inside {\tt PrintTwice()} there is no such thing as {\tt argument}.

Variables like this are said to be {\bf local}.  In order to
keep track of parameters and local variables, it is useful to
draw a {\bf stack diagram}.  Like state diagrams, stack diagrams
show the value of each variable, but the variables are contained
in larger boxes that indicate which function they belong to.

For example, the stack diagram for {\tt PrintTwice()} looks 
like this:

\vspace{0.1in}
\centerline{\epsfig{figure=figs/stack.pdf,width=6cm}}
\vspace{0.1in}
%
Whenever a function is called, it creates a new {\bf instance}
of that function.  Each instance of a function contains the
parameters and local variables for that function.  In the
diagram an instance of a function is represented by a box
with the name of the function on the outside and the variables
and parameters inside.

In the example, {\tt main()} has one local variable, {\tt argument}, and
no parameters.  {\tt PrintTwice()} has no local variables and one
parameter, named {\tt phil}.

\section {Functions with multiple parameters}
\index{parameter!multiple}
\index{function!multiple parameter}
%\index{class!Time}

The syntax for declaring and invoking functions with multiple
parameters is a common source of errors.  First, remember
that you have to declare the type of every parameter.  For
example

\begin{verbatim}
  void PrintTime (int hour, int minute) 
  {
    printf ("%i", hour);
    printf (":");
    printf ("%i", minute);
  }
\end{verbatim}
%
It might be tempting to write {\tt (int hour, minute)}, but
that format is only legal for variable declarations, not
for parameters.

Another common source of confusion is that you do not have
to declare the types of arguments.  The following is wrong!

\begin{verbatim}
    int hour = 11;
    int minute = 59;
    PrintTime (int hour, int minute);   /* WRONG! */
\end{verbatim}
%
In this case, the compiler can tell the type of {\tt hour}
and {\tt minute} by looking at their declarations.  It is
unnecessary and illegal to include the type when you pass them
as arguments.  The correct
syntax is {\tt PrintTime (hour, minute);}.

\section {Functions with results}
\index{fruitful function}
\index{function!fruitful}

You might have noticed by now that some of the functions we are using,
like the math functions, yield results.  Other functions,
like {\tt PrintNewLine}, perform an action but
don't return a value.  That raises some questions:

\begin{itemize}

\item What happens if you call a function and you don't
do anything with the result (i.e. you don't assign it to
a variable or use it as part of a larger expression)?

\item What happens if you use a function without a result as part
of an expression, like {\tt PrintNewLine() + 7}?

\item Can we write functions that yield results, or are we
stuck with things like {\tt PrintNewLine()} and {\tt PrintTwice()}?

\end{itemize}

The answer to the third question is ``yes, you can write functions that
return values,'' and we'll do it in a couple of chapters.  I will
leave it up to you to answer the other two questions by trying them
out.  Any time you have a question about what is legal or
illegal in C, a good way to find out is to ask the compiler.

\section{Glossary}

\begin{description}

\item[constant:] A named storage location similar to a variable, that can not be changed
once it has been initialised.

\item[floating-point:] A type of variable (or value) that can contain
fractions as well as integers.  There are a few floating-point types
in C; the one we use in this book is {\tt double}.

\item[initialization:]  A statement that declares a new variable
and assigns a value to it at the same time.

\item[function:]  A named sequence of statements that performs some
useful function.  Functions may or may not take parameters, and may
or may not produce a result.

\item[parameter:]  A piece of information you provide
in order to call a function.  Parameters are like variables in
the sense that they contain values and have types.

\item[argument:]  A value that you provide when you call a
function.  This value must have the same type as the corresponding
parameter.

\item[call:]  Cause a function to be executed.

\index{floating-point}
\index{function}
\index{parameter}
\index{argument}
\index{call}
\index{initialization}

\end{description}


\section{Exercises}
\setcounter{exercisenum}{0}


% LaTeX source for textbook ``How to think like a computer scientist''
% Copyright (C) 1999  Allen B. Downey
% Copyright (C) 2009  Thomas Scheffler
% Copyright (C) 2023  Michael Penta


\begin{exercise}
	
Evaluate each of the following expressions to determine the resulting value. 
Use the following variables and values when you evaluate the expressions:
\begin{verbatim}
	int x= 2;
	double y = 1.2;
	char z = 'A'
\end{verbatim}
%
\begin{enumerate}
	\item {\tt x + 1}
	\item {\tt x + y}
	\item {\tt x/3}
	\item {\tt x/3.0}
	\item {\tt (int)y}
	\item {\tt (int)z}
\end{enumerate}
\end{exercise}

%%%%%%%%%%%%%%%%%%%%%%%%%%%%%%%%%
\begin{exercise}

The point of this exercise is to practice reading code and to
make sure that you understand the flow of execution through
a program with multiple functions.

\begin{enumerate}

\item What is the output of the following program?  Be precise
about where there are spaces and where there are newlines.

HINT: Start by describing in words what {\tt ping()} and
{\tt baffle()} do when they are invoked.

\begin{verbatim}
 #include <stdio.h>
 #include <stdlib.h>

 void ping(void);
 void baffle(void);
 void zoop(void);

 int main (void)
 {
 	printf ("No, I ");
 	zoop ();
 	printf ("I ");
 	baffle ();
 	return EXIT_SUCCESS;
  }
 void ping(void)
 {
 	printf (".\n"); 
 }
 void baffle(void)
 {
 	printf("wug");
 	ping (); 
 }
 void zoop(void)
 {
 	baffle ();
 	printf ("You wugga ");
 	baffle ();
 }
\end{verbatim}
%

\item Draw a stack diagram that shows the state of the program
the first time {\tt ping()} is invoked.

\end{enumerate}

\end{exercise}

%%%%%%%%%%%%%%%%%%%%%%%%%%%%%%%%%

\begin{exercise}
The point of this exercise is to make sure you understand how
to write and invoke functions that take parameters.

\begin{enumerate}

\item Write a function prototype for a function named {\tt zool()} that
takes three parameters: an {\tt int} and two {\tt char}.

\item Write a line of code that invokes {\tt zool()}, passing
as arguments the value {\tt 11}, the letter {\tt a}, and the letter {\tt z}.
\end{enumerate}


\end{exercise}


%%%%%%%%%%%%%%%%%%%%%%%%%%%%%%%%%%

\begin{exercise}

The purpose of this exercise is to take code from a previous exercise
and encapsulate it in a function that takes parameters.  You should
start with a working solution to exercise
%~\ref{ex.date}.

\begin{enumerate}

\item Write a function definition and prototype for a function called {\tt printDateAmerican()}
that takes the day, month and year as parameters and that
prints them in American format.

\item Test your function by invoking it from {\tt main()} and passing
appropriate arguments.  The output should look something like this
(except that the date might be different):
%
\begin{verbatim}
3/29/2022
\end{verbatim}
%
\item Once you have successfully run the{\tt printDateAmerican()}, write another
function called {\tt printDateEuropean()} that prints the date in
European format.

\item Once you have the two functions working, write a main program that asks the user for the day, month, and year and scan the values into variables.
Call each of your functions with the given input. Do this three times - each time prompt and scan the data from the user and call both functions with the data.

\end{enumerate}

%%%%%%%%%%%%%%%%%%%%%%%%%%%%%%%%%%
\end{exercise}

\begin{exercise}
\label{ex.multadd}

Many computations can be expressed concisely using the ``multadd''
operation, which takes three operands and computes {\tt a*b + c}.  Some
processors even provide a hardware implementation of this operation for
floating-point numbers.

\begin{enumerate}

\item Write a function definition and prototype for a function called {\tt multadd()} that takes three {\tt doubles}
as parameters and that prints their multadditionization.

\item Write a {\tt main()} function that tests {\tt multadd()} by invoking it with a
few simple parameters, like {\tt 1.0, 2.0, 3.0}, and then prints
the result, which should be {\tt 5.0}.

\item Also in {\tt main()}, use {\tt multadd()} to compute the
following value:


{\bf NOTE:} Do not let the math scare you -- you don't have to understand the math to write the code.
Break down each piece. Leverage variables and functions. Look for patterns.

%
\begin{eqnarray*}
& \sin \frac{\pi}{4} + \frac{\cos \frac{\pi}{4}}{2} & 
%\\
%\\
%& \log 10 + \log 20 &
\end{eqnarray*}
%
\item Write a function called {\tt yikes()} that takes a
double as a parameter and that uses {\tt multadd()} to calculate
and print
%
\begin{eqnarray*}
x e^{-x} + \sqrt{1 - e^{-x}}
\end{eqnarray*}
%
HINT: the Math function for raising $e$ to a power is {\tt double exp(double x);}.

\end{enumerate}

In the last part, you get a chance to write a function that invokes
a function you wrote.  Whenever you do that, it is a good idea to
test the first function carefully before you start working on the
second.  Otherwise, you might find yourself debugging two functions
at the same time, which can be very difficult.

One of the purposes of this exercise is to practice pattern-matching:
the ability to recognize a specific problem as an instance of a
general category of problems (think about how these meet the pattern of the multadd function)


\end{exercise}


