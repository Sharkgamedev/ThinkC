% LaTeX source for textbook ``How to think like a computer scientist''
% Copyright (C) 2009  Thomas Scheffler


\chapter{Coding Style}
\section{A short guide on style}
\index{style}
\index{coding style}


In the last few sections, I used the phrase ``by convention''
several times to indicate design decisions that are arbitrary
in the sense that there are no significant reasons to do things
one way or another, but dictated by convention.

In these cases, it is to your advantage to be familiar with
convention and use it, since it will make your programs easier
for others to understand.  At the same time, it is important to
distinguish between (at least) three kinds of rules:

\begin{description}

\item[Divine law:]  This is my phrase to indicate a rule that
is true because of some underlying principle of logic or
mathematics, and that is true in any programming language
(or other formal system).  For example, there is no way to
specify the location and size of a bounding box using fewer
than four pieces of information.  Another example is that adding
integers is commutative.  That's part of the definition of
addition and has nothing to do with C.

\item[Rules of C:]  These are the syntactic and semantic
rules of C that you cannot violate, because the
resulting program will not compile or run.  Some are arbitrary;
for example, the fact that the {\tt =} symbol represents
assignment and {\em not} equality.  Others reflect
underlying limitations of the compilation or execution process.
For example, you have to specify the types of parameters, but
not arguments.

\item[Style and convention:]  There are a lot of rules that
are not enforced by the compiler, but that are essential for
writing programs that are correct, that you can debug and
modify, and that others can read.  Examples include indentation
and the placement of squiggly braces, as well as conventions
for naming variables, functions and types.

\end{description}

In this section I will briefly summarize the coding style used within
this book. It follows loosely the "Nasa C Style Guide" 
\footnote{www.scribd.com/doc/6878959/NASA-C-programming-guide}
and its main
intent is on readability rather than saving space or typing effort.

%/ref 
Since C has such a long history of usage, many 
different coding styles have been developed and used. It is important
that you can read them and follow one particular scheme in all
your code. This makes it much more accessible should you find
yourself in a position where you have to share your work with other 
people or have to access code written by your younger self - many years ago...

\section{Naming conventions and capitalization rules}

As a general rule, you should always choose meaningful names for 
your identifiers. Ideally the name of a variable or function already explains
its behaviour or use.

It may be more typing effort to use a function named  {\tt FindSubString()} 
rather than  {\tt FndSStr()}. However, the former is almost self describing 
and might save you a lot in debugging-time.   

\textbf{Don't use single letter variable names!}

Similarly to functions, you should give your variables names that
speak for themselves and make clear what values will be stored
by this variable.
There are few noticeable exceptions to this rule:
People use {\tt i}, {\tt j} and {\tt k} as counter variables in loops and
for spacial coordinates people use {\tt x}, {\tt y} and {\tt z}.
Use these conventions if they suit you. Don't try to invent new
conventions all by yourself.


The following capitalization style shold be used for the different elements in your
program. The consistent use of one style gives the programmer and the reader
of the source code a quick way to determine the meaning of different items
in your program:


\begin{description}
\item[variableNames: ] variable names always start with lower-case, multiple 
words are separated by capitalizing the first letter. 
\item[CONSTANTS: ] use all upper case letters. In order to avoid name space
collisions it might be necessary to use a prefix such as {\tt MY\_CONSTANT}.
\item[FunctionNames:] start always with upper case and should possibly 
contain a verb describing the function. Names for functions that test values should 
start with '{\tt Is}' or '{\tt Are}'.  
\item[UserDefinedTypes\_t:] always end in '{\tt \_t}'. Type names names must be 
capitalised in order to avoid conflict with POSIX names.
\item[pointerNames\_p:] in order to visually separate pointer variables from
ordinary variables you should consider ending pointers with '{\tt \_p}'.
\end{description}

%%
\section{Bracing style}

There exist different bracing or indent styles that serve the goal
to make your code more readable through the use of a consistent 
indentation for control block structures.
The styles differ in the way the braces are indented with the rest
of the control block.
This book uses the BSD/Allman Style because its is the most 
readable of the four. It needs more horizontal space than the K\&R Style
but it makes it very easy to track opening and closing braces.

When you are writing programs, make sure that you are using one
style consistently. In larger projects all contributors should agree
on the style they are using. Modern programming environments like
Eclipse support you through the automatic enforcement of a single style.

\begin{verbatim}
/*Whitesmiths Style*/
   if (condition)            
       {
       statement1; 
       statement2;
       }
\end{verbatim}

Is named after Whitesmiths C, an early commercial C compiler that 
used this style in its examples. Some people refer to it as the 
One True Brace Style.

\begin{verbatim}

/*GNU Style*/
   if (condition)
     {
       statement1;
       statement2;
     }
\end{verbatim}

Indents are always four spaces per level, with the braces halfway between the outer and inner indent levels.

\begin{verbatim}

/*K&R/Kernel Style*/
   if (condition) {
       statement1;
       statement2;
   }
\end{verbatim}

This style is named after the programming examples in the book
\emph{The C Programming Language} by Brian W. Kernighan and 
Dennis Ritchie (the C inventors). 

The K\&R style is the style that is hardest to read. 
The opening brace happens to be at the far right side of the control statement
and can be hard to find. The braces therefore have different indentation levels.
Nevertheless, many C programs use this style. So you should be able 
to read it.

\begin{verbatim}

/*BSD/Allman Style*/
   if (condition)
   {
       statement1;
       statement2;
   }
\end{verbatim}

This style is used for all the examples in this book.

\section{Layout}

Block comments should be used at the top of your file, before all
function declarations, to explain the purpose of the program and give additional
information.

You should also use a similar documentation style before every 
relevant function in your program.
\begin{verbatim}

/*
 * File:     test.c
 * Author:   Peter Programmer
 * Date:     May, 29th, 2009
 *
 * Purpose: to demonstrate good programming
 *          practise
 * /

#include <stdlib.h>

/*
 * main function, does not use arguments
 */

int main (void)
{
    return EXIT_SUCCESS;
}
 
\end{verbatim}



