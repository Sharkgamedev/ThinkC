% LaTeX source for textbook ``How to think like a computer scientist''
% Copyright (C) 1999  Allen B. Downey
% Copyright (C) 2009  Thomas Scheffler


\chapter{Arrays}
\label{arrays}
\index{arrays}
\index{type!array}

A {\bf array} is a set of values where each value is identified and referenced by a
number (called an index).  The nice thing
about arrays is that they can be made up of any type of element,
including basic types like {\tt int}s and {\tt double}s, 
but all the values in an array have to have the same type.
%and user-defined types like {\tt Point} and {\tt Time}.


When you declare an array, you have to determine the number of
elements in the array. Otherwise the declaration looks similar to other variable types:

\begin{verbatim}
    int c[4];
    double values[10];
\end{verbatim}


%

Syntactically, array variables look like other C variables except that they are followed 
by {\tt [NUMBER\_OF\_ELEMENTS]}, the number of elements in the array enclosed in square brackets. 
The first line in our example, {\tt int c[4];} is of the type "array of integers" and creates a array of four integers named {\tt c}.
The second line, {\tt double values[10];} has the type "array of doubles" and
  creates an array of 10 {\tt double}s. 

%The number
%of elements in {\tt values} depends on {\tt size}. You can use any
%integer expression to determine the size of an array.
%!!! Not in C
%this would be dynamic arrays, that can not be initialised at definition time!!!

%

C allows you to to initialize the element values of an array immediately
after you have declared it.  The values  for the individual elements must be 
enclosed in curly brackets {\tt \{\}} and separated by comma, as in the following example:

\begin{verbatim}
    int c[4] = {0, 0, 0, 0};
\end{verbatim}

This statement creates an array of four elements and initializes
all of them to zero.
This syntax is only legal at initialisation time. Later in your program you can only
assign values for the array element by element.

%
The following figure shows how arrays are represented in state
diagrams:

%\myfig{figure=figs/array.eps}

\unitlength0.1cm

\begin{picture}(40,10)(-30,-5)
%\put(-4,1.5){{\Large \texttt{c}}}
%\put(0,1.5){\framebox(2,2)}
%\thicklines
%\put(2,2.5){\vector(1,0){8}}
%\thinlines
\put(5,1.5){{\Large \texttt{c}}}
\put(10,0){\framebox(7,5){\textbf{\textsf{0}}}}
\put(17,0){\framebox(7,5){\textbf{\textsf{0}}}}
\put(24,0){\framebox(7,5){\textbf{\textsf{0}}}}
\put(31,0){\framebox(7,5){\textbf{\textsf{0}}}}

\put(10.5,-4){{\scriptsize \texttt{c[0]}}}
\put(17.5,-4){{\scriptsize \texttt{c[1]}}}
\put(24.5,-4){{\scriptsize \texttt{c[2]}}}
\put(31.5,-4){{\scriptsize \texttt{c[3]}}}

\end{picture}

The large numbers inside the boxes are the values of the {\bf elements} in
the array.  The small numbers outside the boxes are the
indices used to identify each box.  When you allocate a new
array, without initializing, the arrays elements typically
contain arbitrary values and you must initialize them to
a meaningful value before using them.

%%
\section{Accessing elements}
\index{element}
\index{array!element}

The {\tt []} operator allows us to read and write the individual elements of an array.  
The indices start at zero, so {\tt c[0]}
refers to the first element of the array, and {\tt c[1]}
refers to the second element.  You can use the {\tt []} operator
anywhere in an expression:


\begin{verbatim}
    c[0] = 7;
    c[1] = c[0] * 2;
    c[2]++;
    c[3] -= 60;
\end{verbatim}
%
All of these are legal assignment statements.  Here is the
effect of this code fragment:


%\myfig{figure=figs/array2.eps}

\unitlength0.1cm

\begin{picture}(40,10)(-30,-5)
%\put(-11,1.5){\texttt{count}}
%\put(0,1.5){\framebox(2,2)}
%\thicklines
%\put(2,2.5){\vector(1,0){8}}
%\thinlines
\put(5,1.5){{\Large \texttt{c}}}

\put(10,0){\framebox(7,5){\textbf{\textsf{7}}}}
\put(17,0){\framebox(7,5){\textbf{\textsf{14}}}}
\put(24,0){\framebox(7,5){\textbf{\textsf{1}}}}
\put(31,0){\framebox(7,5){\textbf{\textsf{-60}}}}

\put(10.5,-4){{\scriptsize \texttt{c[0]}}}
\put(17.5,-4){{\scriptsize \texttt{c[1]}}}
\put(24.5,-4){{\scriptsize \texttt{c[2]}}}
\put(31.5,-4){{\scriptsize \texttt{c[3]}}}

\end{picture}

By now you should have noticed that the four elements of this array
are numbered from 0 to 3, which means that there is no element with
the index 4.  

Nevertheless, it is a common error to go
beyond the bounds of an array. In safer languages such as Java, this will cause an 
error and most likely the program quits. C does not check array boundaries, so
your program can go on accessing memory locations beyond the array itself, as if 
they where part of the array. This is most likely wrong and can cause very
severe bugs in your program. 
\begin{quote}

{\bf It is necessary that you, as a programmer,
make sure that your code correctly observes array boundaries!}

\end{quote}




\index{run-time error}
\index{index}
\index{expression}

You can use any expression as an index, as long as it has type {\tt
int}.  One of the most common ways to index an array is with a loop
variable.  For example:

\begin{verbatim}
    int i = 0;
    while (i < 4) 
    {
        printf ("%i\n", c[i]);
        i++;
    }
\end{verbatim}

%
This is a standard {\tt while} loop that counts from 0
up to 4, and when the loop variable {\tt i} is 4, the
condition fails and the loop terminates.  Thus, the body
of the loop is only executed when {\tt i} is 0, 1, 2 and 3.

\index{loop}
\index{loop variable}
\index{variable!loop}

Each time through the loop we use {\tt i} as an index into
the array, printing the {\tt i}th element.  This type of
array traversal is very common.  Arrays and loops go together
like fava beans and a nice Chianti.

\index{fava beans}
\index{Chianti}

%%
\section{Copying arrays}
\index{array!copying}

Arrays can be a very convenient solution for a number of problems, like
storing and processing large sets of data.

However, there is very little that C does automatically for you. For example
you can not set all the elements of an array at the same time and you can not
assign one array to the other, even if they are identical in type and number of elements.


\begin{verbatim}
    double a[3] = {1.0, 1.0, 1.0};
    double b[3];

    a = 0.0;     /* Wrong! */
    b = a;       /* Wrong! */
\end{verbatim}
%

In order to set all of the elements of an array to some value, you must do so element by element.
To copy the contents of one array to another, you must again do so, by copying each element from
one array to the other.

\begin{verbatim}
    int i = 0;
    while (i < 3) 
    {
        b[i] = a[i];
        i++;
    }
\end{verbatim}

%%
\section{{\tt for} loops}

The loops we have written so far have a number of elements
in common.  All of them start by initializing a variable;
they have a test, or condition, that depends on that variable;
and inside the loop they do something to that variable,
like increment it.

\index{loop!for}
\index{for}
\index{statement!for}

This type of loop is so common that there is an alternate
loop statement, called {\tt for}, that expresses it more
concisely.  The general syntax looks like this:

\begin{verbatim}
    for (INITIALIZER; CONDITION; INCREMENTOR) 
    {
        BODY
    }
\end{verbatim}
%
This statement is exactly equivalent to

\begin{verbatim}
    INITIALIZER;
    while (CONDITION) 
    {
        BODY
        INCREMENTOR
    }
\end{verbatim}
%
except that it is more concise and, since it puts all the
loop-related statements in one place, it is easier to read.
For example:

\begin{verbatim}
    int i;
    for (i = 0; i < 4; i++) 
    {
        printf("%i\n", c[i]);
    }
\end{verbatim}
%
is equivalent to 

\begin{verbatim}
    int i = 0;
    while (i < 4) 
    {
        printf("%i\n", c[i]);
        i++;
    }
\end{verbatim}

%%
\section{Array length}
\label{Array length}
\index{length!array}
\index{array!length}
\index{sizeof}
\index{operator!sizeof}

C does not provide us with a convenient way to determine the
actual length of an array. Knowing the size of an array would
be convenient when we are looping through all elements of
the array and need to stop with the last element.

In order to determine the array length we could use the {\tt sizeof()} 
operator, that calculates the size of data types in bytes.
Most data types in C use more than one byte to store their values,
therefore it becomes necessary to divide the byte-count for the array by 
the byte-count for a single element to establish the number of elements
in the array.
\begin{verbatim}
    sizeof(ARRAY)/sizeof(ARRAY_ELEMENT)
\end{verbatim}

It is a good idea to use this value as the upper bound of a loop,
rather than a constant.  That way, if the size of the array
changes, you won't have to go through the program changing all the
loops; they will work correctly for any size array.

\begin{verbatim}
    int i, length;
    length = sizeof (c) / sizeof (c[0]);

    for (i = 0; i < length; i++) 
    {
        printf("%i\n", c[i]);
    }
\end{verbatim}
%
The last time the body of the loop gets executed, the value of {\tt i}
is {\tt length - 1}, which is the index of the last element.  When
{\tt i} is equal to {\tt length}, the condition fails and the body
is not executed, which is a good thing, since it would access a
memory location that is not part of the array.


%%
\pagebreak
\section{Arrays and random}
\index{statistics}
\index{distribution}
\index{mean}

The numbers generated by {\tt rand()} are supposed to be distributed
uniformly.  That means that each value in the range should be
equally likely.  If we count the number of times each value appears,
it should be roughly the same for all values, provided that we
generate a large number of values.

In the next few sections, we will write programs that generate
a sequence of random numbers and check whether this property
holds true.

%%
\section{Array of random numbers}
\label{Array of random numbers}

The first step is to generate a large number of random values
and store them in a array.  By ``large number,'' of course,
I mean 20.  It's always a good idea to start with a manageable
number, to help with debugging, and then increase it later.

The following function takes three arguments, an array of integers, 
the size of the array and an upper bound for the random values.  
It fills the array of {\tt int}s with random values between 0 and {\tt upperBound-1}.

\begin{verbatim}
    void RandomizeArray (int array[], int length, int upperBound) 
    {
        int i;
        for (i = 0; i < length; i++) 
        {
            array[i] = rand() % upperBound;
        }
    }
\end{verbatim}
%
The return type is {\tt void}, which means that
this function does not return any value to the calling function.
To test this function, it is convenient to have a function that
outputs the contents of a array.

\begin{verbatim}
    void PrintArray (int array[], int length) 
    {
        int i;
        for (i = 0; i < length; i++) 
        {
            printf ("%i ",  array[i]);
        }
    }
\end{verbatim}
%
The following code generates an array filled with random values and outputs it:

\begin{verbatim}
    int r_array[20];
    int upperBound = 10;
    int length = sizeof(r_array) / sizeof(r_array[0]);
  
    RandomizeArray (r_array, length, upperBound);
    PrintArray (r_array, length);
\end{verbatim}

%
On my machine the output is:

\begin{verbatim}
3 6 7 5 3 5 6 2 9 1 2 7 0 9 3 6 0 6 2 6 
\end{verbatim}
\nopagebreak%
which is pretty random-looking.  Your results may differ.

If these numbers are really random,
we expect each digit to appear the same number of times---twice
each.  In fact, the number 6 appears five times, and the numbers 4
and 8 never appear at all.

Do these results mean the values are not really uniform?  It's
hard to tell.  With so few values, the chances are slim
that we would get exactly what we expect.  But as the number
of values increases, the outcome should be more predictable.

To test this theory, we'll write some programs that count the
number of times each value appears, and then see what happens
when we increase the number of elements in our array.

%%
\section{Passing an array to a function}
\label{Passing an array to a function}
\index{call by reference}
\index{call by value}
\index{array parameters}
You probably have noticed that our {\tt RandomizeArray()} function 
looked a bit unusual. We pass an array to this function and expect 
to get a a randomized array back. Nevertheless, we have declared it to 
be a {\tt void} function, and miraculously the function appears to have 
altered the array.

This behaviour goes against everything what I have said about the
use of variables in functions so far.
C typically uses the so called {\bf call-by-value} evaluation of
expressions. If you pass a value to a function it gets copied from
the calling function to a variable in the called function. The same
is true if the function returns a value.
Changes to the internal variable in the called function do not affect the external 
values of the calling function.

When we pass an array to a function this behaviour changes to
something called {\bf call-by-reference} evaluation.
C does not copy the array to an internal array --  it rather generates a
reference to the original array and any operation in the called function 
directly affects the original array.
This is also the reason why we do not have to return anything from our 
function. The changes have already taken place. 

Call by reference also makes it necessary to supply the length of
the array to the called function, since invoking  the {\tt sizeof}
operator in the called function would determine the size of the reference
and not the original array.

%!!!Reference to later chapter needed!!!
We will further discuss call by reference and call by value in 
Section~\ref{Pointers and Addresses}, Section~\ref{Call by value} and
\ref{Call by reference}.

%%
\section{Counting}
\label{counting}
\index{traverse!counting}
\index{loop!counting}
\index{counter}

A good approach to problems like this is to think of simple functions
that are easy to write, and that might turn out to be useful.  Then
you can combine them into a solution.  This approach is sometimes
called {\bf bottom-up design}.  

Of course, it is not easy to
know ahead of time which functions are likely to be useful, but as you
gain experience you will have a better idea.
\index{bottom-up design}
\index{program development!bottom-up}
Also, it is not always obvious what sort of things are easy to write,
but a good approach is to look for subproblems that fit a pattern you
have seen before.

\index{pattern!counter}

%Back in Section~\ref{loopcount} we looked at a loop that traversed a
%string and counted the number of times a given letter appeared.  
In our current example we want to examine a potentially large set
of elements and count the number of times a certain value appears.
You
can think of this program as an example of a pattern called ``traverse
and count.''  The elements of this pattern are:

\begin{itemize}

\item A set or container that can be traversed, like a string
or a array.

\item A test that you can apply to each element in the container.

\item A counter that keeps track of how many elements pass
the test.

\end{itemize}

In this case, I have a function in mind called {\tt HowMany()} that
counts the number of elements in a array that are equal to a given value.
The parameters are the array, the length of the array and the integer value we are looking
for.  The return value is the number of times the value appears.

\begin{verbatim}
    int HowMany (int array[], int length, int value) 
    {
        int i; 
        int count = 0;
  
        for (i=0; i < length; i++) 
            {
                if (array[i] == value) count++;
            }
        return count;
    }
\end{verbatim}


\section{Checking the other values}

{\tt HowMany()} only counts the occurrences of a particular value, and
we are interested in seeing how many times each value appears.
We can solve that problem with a loop:

\begin{verbatim}
    int i;
    int r_array[20];
    int upperBound = 10;
    int length = sizeof(r_array) / sizeof(r_array[0]);
  
    RandomizeArray(r_array, length, upperBound);

    printf ("value\tHowMany\n");
    for (i = 0; i < upperBound; i++) 
    {
        printf("%i\t%i\n", i, HowMany(r_array, length, i));
    }
\end{verbatim}
%

%! ! ! Applies only to C++! ! !
%Notice that it is legal to declare a variable inside a {\tt for}
%statement.  This syntax is sometimes convenient, but you should
%be aware that a variable declared inside a loop only exists
%inside the loop.  If you try to refer to {\tt i} later, you
%will get a compiler error.

This code uses the loop variable as an argument to
{\tt HowMany()}, in order to check each value between 0 and 9,
in order.  The result is:

\begin{verbatim}
value   HowMany
0       2
1       1
2       3
3       3
4       0
5       2
6       5
7       2
8       0
9       2
\end{verbatim}
%
Again, it is hard to tell if the digits are really appearing
equally often.  If we increase the size of the array to 100,000 we
get the following:

\begin{verbatim}
value   HowMany
0       10130
1       10072
2       9990
3       9842
4       10174
5       9930
6       10059
7       9954
8       9891
9       9958
\end{verbatim}
%
In each case, the number of appearances is within about 1\% of
the expected value (10,000), so we conclude that the random
numbers are probably uniform.

\section {A histogram}
\index{histogram}

It is often useful to take the data from the previous tables
and store them for later access, rather than just print them.
What we need is a way to store 10 integers.  We could create
10 integer variables with names like {\tt howManyOnes},
{\tt howManyTwos}, etc.  But that would require a lot of
typing, and it would be a real pain later if we decided to
change the range of values.

A better solution is to use a array with length 10.  That
way we can create all ten storage locations at once and we
can access them using indices, rather than ten different names.
Here's how:

\begin{verbatim}
    int i;
    int upperBound = 10;
    int r_array[100000];
    int histogram[upperBound];
    int r_array_length = sizeof(r_array) / sizeof(r_array[0]);
  
    RandomizeArray(r_array, r_array_length, upperBound);

    for (i = 0; i < upperBound; i++) 
    {
        int count = HowMany(r_array, length, i);
        histogram[i] = count;
    }  
\end{verbatim}
%
I called the array {\bf histogram} because that's
a statistical term for a array of numbers that counts the
number of appearances of a range of values.

\index{histogram}

The tricky thing here is that I am using the loop variable
in two different ways.  First, it is an argument to {\tt HowMany()},
specifying which value I am interested in.  Second, it is
an index into the histogram, specifying which location I should
store the result in.

\section{A single-pass solution}

Although this code works, it is not as efficient as it could
be.  Every time it calls {\tt HowMany()}, it traverses the
entire array.  In this example we have to traverse the
array ten times!

It would be better to make a single pass through the array.
For each value in the array we could find the corresponding
counter and increment it.  In other words, we can use the
value from the array as an index into the histogram.  Here's
what that looks like:

\begin{verbatim}
    int upperBound = 10;
    int histogram[upperBound] = {0, 0, 0, 0, 0, 0, 0, 0, 0, 0};

    for (i = 0; i < r_array_length; i++) 
    {
        int index = r_array[i];
        histogram[index]++;
    }
\end{verbatim}
%
The second line initializes the elements of the histogram to
zeroes.  That way, when we use the increment
operator ({\tt ++}) inside the loop, we know we are starting from zero.
Forgetting to initialize counters is a common error.

As an exercise, encapsulate this code in a function called {\tt Histogram()} 
that takes an array and the range of values in the array
(in this case 0 through 10) as two parameters \texttt{min} and \texttt{max}. 
You should pass a second array to the function where a histogram of the
values in the array can be stored.


\section{Partially filled arrays}
In C, an array's sized is fixed at the moment it is declared. But what if we don't know how much space is needed when an array is declared. For example, we may want to store the names of students who enrolled in a course, but we don't know how many students there will be until they register. Or we may want to store some numbers that are entered by the user, but we don't know how many numbers there will be until they stop entering them. In these cases, we can use a {\bf partially filled array}.

A partially filled array is an array that has some empty slots that are not used for storing data. For example, if we declare an array of size 10, but only use 5 elements to store some data, then we have a partially filled array with 5 empty slots. The advantage of using a partially filled array is that we can allocate more space than we need at first, and then fill it up later as needed. The disadvantage is that we may waste some memory if we allocate too much space than we need.

To work with partially filled arrays in C, we need two things: a variable that defines the maximum size of the array (its capacity), and a local variable that keeps track of how many elements are actually used in the array (usually called size). The size variable also allows us to ensure that we don't exceed the capacity of the array when adding new elements. All the values are stored contiguously in the array starting at the zero index. This means we can use the size variable to find the next open location. We can assign a value at that location, then increase the size.

\begin{verbatim}
 //the maximum size of the array
 #define CAPACITY  10
 
 // Declare an int array with the max capacity 
 int arr[CAPACITY];

 // Declare a variable to keep track of how many elements are used
 int size = 0;

 //get some values from the user using a user-terminated loop (or sentinel loop)

 // Input from user
 int num;
 //Prompt and scan for the initial values
 puts("Enter some numbers (enter -1 to stop):");
 scanf("%d", &num);

 //while the user enters a valid number 
 //and we have room in the array

 while (num != -1 && size < CAPACITY) 
 {
	  arr[size] = num; 	// Add num to arr at index size
	  size++;  // Increment size by 1
	
	  //Prompt and scan for the next values
	  puts("Enter some numbers (enter -1 to stop):");
	  scanf("%d", &num); // Read another number from input
 }

\end{verbatim}
%
The second line initializes the elements of the histogram to
zeroes.  That way, when we use the increment
operator ({\tt ++}) inside the loop, we know we are starting from zero.
Forgetting to initialize counters is a common error.

As an exercise, encapsulate this code in a function called {\tt Histogram()} 
that takes an array and the range of values in the array
(in this case 0 through 10) as two parameters \texttt{min} and \texttt{max}. 
You should pass a second array to the function where a histogram of the
values in the array can be stored.

\section{Glossary}

\begin{description}

\item[array:]  A named collection of values, where all the
values have the same type, and each value is identified by
an index.

\item[element:]  One of the values in a array.  The {\tt []}
operator selects elements of a array.

\item[index:]  An integer variable or value used to indicate
an element of a array.

\item[increment:]  Increase the value of a variable by one.
The increment operator in C is {\tt ++}. 

\item[decrement:]  Decrease the value of a variable by one.
The decrement operator in C is {\tt -}{\tt -}.

\item[bottom-up design:]  A method of program development that
starts by writing small, useful functions and then assembling
them into larger solutions.

\item[histogram:]  A array of integers where each integer
counts the number of values that fall into a certain range.

\index{array}
\index{element}
\index{index}
\index{deterministic}
\index{histogram}

\end{description}

%%
\section{Exercises}
\setcounter{exercisenum}{0}


% LaTeX source for textbook ``How to think like a computer scientist''
% Copyright (C) 1999  Allen B. Downey
% Copyright (C) 2009  Thomas Scheffler

%%%%%%%%%%%%%%%%%%%%%%%%%%%%%%%%%%%%%%
\begin{exercise}
	
	It helpful to have a function that prints the contents of an array. Define a function and prototype for a function called printIntArray that has 2 parameters, an int for the size of the array and an array of integers. Use a loop to print the contents of the array in a nicely formatted way. Write a main program that defines two arrays and use the function to print the arrays.  Below is an example output of the program:
	
	\begin{verbatim}
		[ 2, 5, 7, 8, 1 ]
		[ 1, 2, 9, 1 ]
	\end{verbatim}
	
\end{exercise}

%%%%%%%%%%%%%%%%%%%%%%%%%%%%%%%%%%%%%%
\begin{exercise}

A friend of yours shows you the following method and
explains that if {\tt number} is any two-digit number, the program
will output the number backwards.  He claims that if {\tt number} is
17, the method will output {\tt 71}.

Is he right?  If not, explain what the program actually does and
modify it so that it does the right thing.

\begin{verbatim}
   #include <stdio.h>
   #include<stdlib.h>
   
   int main (void)
   {
     int number = 71;
     int lastDigit = number%10;
     int firstDigit = number/10;
     printf("%i",lastDigit + firstDigit);
     return EXIT_SUCCESS;
  }

\end{verbatim}
%%%%%%%%%%%%%%%%%%%%%%%%%%%%%%%%%%%%%%

\end{exercise}
\begin{exercise}
	
	Rewrite the previous program so that it prompts the user for a three digit integer. Create a 3 element array and store each digit into the array. Use your printIntArray function to print the content of the array.
\end{exercise}


%%%%%%%%%%%%%%%%%%%%%%%%%%%%%%%%%%%%%%
\begin{exercise}
Write a function and prototype that takes an array of integers, the length of the array and an integer named
{\tt target} as arguments. 

The function should search through the provided array and should return the first index where
{\tt target} appears in the array, if it does. If {\tt target} is not in the array the function should 
return an invalid index value to indicate an error condition  (e.g.  -1).

Write a main program to test the function. Be sure to print the array as well.
\end{exercise}

%%%%%%%%%%%%%%%%%%%%%%%%%%%%%%%%%%%%%%

%%%%%%%%%%%%%%%%%%%%%%%%%%%%%%%%%%%%%%
\begin{exercise}
	Write a function and prototype that takes an array of integers, and the length of the array.
	
	The function should sum all the values in the array and then return the sum of the array. If the array is empty, return 0.
	
	Write a main program that creates a 20 element array, and uses a user-terminated loop to ask the user to fill the array with positive numbers. The user can stop the loop by entering a negative number. Use the array, call your function and print the resulting sum. Be sure to print the array as well/
\end{exercise}

%%%%%%%%%%%%%%%%%%%%%%%%%%%%%%%%%%%%%%
\begin{exercise}

One not-very-efficient way to sort the elements of an array
is to find the largest element and swap it with the first
element, then find the second-largest element and swap it with
the second, and so on.

\begin{enumerate}

\item Write a function called {\tt IndexOfMaxInRange()} that 
takes an array of integers, finds the
largest element in the given range, and returns its {\em index}.

\item Write a function called {\tt SwapElement()} that takes an
array of integers and two indices, and that swaps the elements
at the given indices.

\item Write a function called {\tt SortArray()} that takes an array of
integers and that uses {\tt IndexOfMaxInRange()} and {\tt SwapElement()}
to sort the array from largest to smallest.

\end{enumerate}
\end{exercise}


%%%%%%%%%%%%%%%%%%%%%%%%%%%%%%%%%%%%%%




