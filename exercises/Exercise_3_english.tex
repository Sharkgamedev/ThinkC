% LaTeX source for textbook ``How to think like a computer scientist''
% Copyright (C) 1999  Allen B. Downey
% Copyright (C) 2009  Thomas Scheffler
% Copyright (C) 2023  Michael Penta


\begin{exercise}
	
Evaluate each of the following expressions to determine the resulting value. 
Use the following variables and values when you evaluate the expressions:
\begin{verbatim}
	int x= 2;
	double y = 1.2;
	char z = 'A'
\end{verbatim}
%
\begin{enumerate}
	\item {\tt x + 1}
	\item {\tt x + y}
	\item {\tt x/3}
	\item {\tt x/3.0}
	\item {\tt (int)y}
	\item {\tt (int)z}
\end{enumerate}
\end{exercise}

%%%%%%%%%%%%%%%%%%%%%%%%%%%%%%%%%
\begin{exercise}

The point of this exercise is to practice reading code and to
make sure that you understand the flow of execution through
a program with multiple functions.

\begin{enumerate}

\item What is the output of the following program?  Be precise
about where there are spaces and where there are newlines.

HINT: Start by describing in words what {\tt ping()} and
{\tt baffle()} do when they are invoked.

\begin{verbatim}
 #include <stdio.h>
 #include <stdlib.h>

 void ping(void);
 void baffle(void);
 void zoop(void);

 int main (void)
 {
 	printf ("No, I ");
 	zoop ();
 	printf ("I ");
 	baffle ();
 	return EXIT_SUCCESS;
  }
 void ping(void)
 {
 	printf (".\n"); 
 }
 void baffle(void)
 {
 	printf("wug");
 	ping (); 
 }
 void zoop(void)
 {
 	baffle ();
 	printf ("You wugga ");
 	baffle ();
 }
\end{verbatim}
%

\item Draw a stack diagram that shows the state of the program
the first time {\tt ping()} is invoked.

\end{enumerate}

\end{exercise}

%%%%%%%%%%%%%%%%%%%%%%%%%%%%%%%%%

\begin{exercise}
The point of this exercise is to make sure you understand how
to write and invoke functions that take parameters.

\begin{enumerate}

\item Write a function prototype for a function named {\tt zool()} that
takes three parameters: an {\tt int} and two {\tt char}.

\item Write a line of code that invokes {\tt zool()}, passing
as arguments the value {\tt 11}, the letter {\tt a}, and the letter {\tt z}.
\end{enumerate}


\end{exercise}


%%%%%%%%%%%%%%%%%%%%%%%%%%%%%%%%%%

\begin{exercise}

The purpose of this exercise is to take code from a previous exercise
and encapsulate it in a function that takes parameters.  You should
start with a working solution to exercise
%~\ref{ex.date}.

\begin{enumerate}

\item Write a function definition and prototype for a function called {\tt printDateAmerican()}
that takes the day, month and year as parameters and that
prints them in American format.

\item Test your function by invoking it from {\tt main()} and passing
appropriate arguments.  The output should look something like this
(except that the date might be different):
%
\begin{verbatim}
3/29/2022
\end{verbatim}
%
\item Once you have successfully run the{\tt printDateAmerican()}, write another
function called {\tt printDateEuropean()} that prints the date in
European format.

\item Once you have the two functions working, write a main program that asks the user for the day, month, and year and scan the values into variables.
Call each of your functions with the given input. Do this three times - each time prompt and scan the data from the user and call both functions with the data.

\end{enumerate}

%%%%%%%%%%%%%%%%%%%%%%%%%%%%%%%%%%
\end{exercise}

\begin{exercise}
\label{ex.multadd}

Many computations can be expressed concisely using the ``multadd''
operation, which takes three operands and computes {\tt a*b + c}.  Some
processors even provide a hardware implementation of this operation for
floating-point numbers.

\begin{enumerate}

\item Write a function definition and prototype for a function called {\tt multadd()} that takes three {\tt doubles}
as parameters and that prints their multadditionization.

\item Write a {\tt main()} function that tests {\tt multadd()} by invoking it with a
few simple parameters, like {\tt 1.0, 2.0, 3.0}, and then prints
the result, which should be {\tt 5.0}.

\item Also in {\tt main()}, use {\tt multadd()} to compute the
following value:


{\bf NOTE:} Do not let the math scare you -- you don't have to understand the math to write the code.
Break down each piece. Leverage variables and functions. Look for patterns.

%
\begin{eqnarray*}
& \sin \frac{\pi}{4} + \frac{\cos \frac{\pi}{4}}{2} & 
%\\
%\\
%& \log 10 + \log 20 &
\end{eqnarray*}
%
\item Write a function called {\tt yikes()} that takes a
double as a parameter and that uses {\tt multadd()} to calculate
and print
%
\begin{eqnarray*}
x e^{-x} + \sqrt{1 - e^{-x}}
\end{eqnarray*}
%
HINT: the Math function for raising $e$ to a power is {\tt double exp(double x);}.

\end{enumerate}

In the last part, you get a chance to write a function that invokes
a function you wrote.  Whenever you do that, it is a good idea to
test the first function carefully before you start working on the
second.  Otherwise, you might find yourself debugging two functions
at the same time, which can be very difficult.

One of the purposes of this exercise is to practice pattern-matching:
the ability to recognize a specific problem as an instance of a
general category of problems (think about how these meet the pattern of the multadd function)


\end{exercise}