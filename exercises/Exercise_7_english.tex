% LaTeX source for textbook ``How to think like a computer scientist''
% Copyright (C) 1999  Allen B. Downey
% Copyright (C) 2009  Thomas Scheffler

%%%%%%%%%%%%%%%%%%%%%%%%%%%%%%%%%%%%%%
\begin{exercise}
	
	It helpful to have a function that prints the contents of an array. Define a function and prototype for a function called printIntArray that has 2 parameters, an int for the size of the array and an array of integers. Use a loop to print the contents of the array in a nicely formatted way. Write a main program that defines two arrays and use the function to print the arrays.  Below is an example output of the program:
	
	\begin{verbatim}
		[ 2, 5, 7, 8, 1 ]
		[ 1, 2, 9, 1 ]
	\end{verbatim}
	
\end{exercise}

%%%%%%%%%%%%%%%%%%%%%%%%%%%%%%%%%%%%%%
\begin{exercise}

A friend of yours shows you the following method and
explains that if {\tt number} is any two-digit number, the program
will output the number backwards.  He claims that if {\tt number} is
17, the method will output {\tt 71}.

Is he right?  If not, explain what the program actually does and
modify it so that it does the right thing.

\begin{verbatim}
   #include <stdio.h>
   #include<stdlib.h>
   
   int main (void)
   {
     int number = 71;
     int lastDigit = number%10;
     int firstDigit = number/10;
     printf("%i",lastDigit + firstDigit);
     return EXIT_SUCCESS;
  }

\end{verbatim}
%%%%%%%%%%%%%%%%%%%%%%%%%%%%%%%%%%%%%%

\end{exercise}
\begin{exercise}
	
	Rewrite the previous program so that it prompts the user for a three digit integer. Create a 3 element array and store each digit into the array. Use your printIntArray function to print the content of the array.
\end{exercise}


%%%%%%%%%%%%%%%%%%%%%%%%%%%%%%%%%%%%%%
\begin{exercise}
Write a function and prototype that takes an array of integers, the length of the array and an integer named
{\tt target} as arguments. 

The function should search through the provided array and should return the first index where
{\tt target} appears in the array, if it does. If {\tt target} is not in the array the function should 
return an invalid index value to indicate an error condition  (e.g.  -1).
\end{exercise}

%%%%%%%%%%%%%%%%%%%%%%%%%%%%%%%%%%%%%%

%%%%%%%%%%%%%%%%%%%%%%%%%%%%%%%%%%%%%%
\begin{exercise}
	Write a function and prototype that takes an array of integers, and the length of the array.
	
	The function should sum all the values in the array and then return the sum of the array. If the array is empty, return 0.
\end{exercise}

%%%%%%%%%%%%%%%%%%%%%%%%%%%%%%%%%%%%%%
\begin{exercise}

One not-very-efficient way to sort the elements of an array
is to find the largest element and swap it with the first
element, then find the second-largest element and swap it with
the second, and so on.

\begin{enumerate}

\item Write a function called {\tt IndexOfMaxInRange()} that 
takes an array of integers, finds the
largest element in the given range, and returns its {\em index}.

\item Write a function called {\tt SwapElement()} that takes an
array of integers and two indices, and that swaps the elements
at the given indices.

\item Write a function called {\tt SortArray()} that takes an array of
integers and that uses {\tt IndexOfMaxInRange()} and {\tt SwapElement()}
to sort the array from largest to smallest.

\end{enumerate}
\end{exercise}


%%%%%%%%%%%%%%%%%%%%%%%%%%%%%%%%%%%%%%

