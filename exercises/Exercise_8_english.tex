% LaTeX source for textbook ``How to think like a computer scientist''
% Copyright (C) 1999  Allen B. Downey
% Copyright (C) 2009  Thomas Scheffler

%%%%%%%%%%%%%%%%%%%%%%%%%%%%%%%%%%%%%%
\begin{exercise}
A word is said to be ``abecedarian'' if the letters in the
word appear in alphabetical order.  For example, the following
are all 6-letter English abecedarian words.

\begin {quote}
abdest, acknow, acorsy, adempt, adipsy, agnosy, befist, behint,
beknow, bijoux, biopsy, cestuy, chintz, deflux, dehors, dehort,
deinos, diluvy, dimpsy
\end{quote}

\begin{enumerate}

\item Describe an algorithm for checking whether a given word (String)
is abecedarian, assuming that the word contains only lower-case
letters.  Your algorithm can be iterative or recursive.

\item Implement your algorithm in a function called {\tt IsAbecedarian()}.

\end{enumerate}
\end{exercise}

%%%%%%%%%%%%%%%%%%%%%%%%%%%%%%%%%%%%%%


\begin{exercise}
Write a function called {\tt LetterHist()} that takes a string and and an int array as
parameters. The function builds a histogram by filling the count of each letter in the string to the array.
The zeroeth element of the histogram array should contain the number of a's
in the String (upper and lower case); the 25th element should contain
the number of z's.
Your solution should only traverse the String once.
\end{exercise}


%%%%%%%%%%%%%%%%%%%%%%%%%%%%%%%%%%%%%%


\begin{exercise}
A word is said to be a ``doubloon'' if every letter that appears in the
word appears exactly twice.  For example, the following are all the
doubloons I found in my dictionary.

\begin {quote}
Abba, Anna, appall, appearer, appeases, arraigning, beriberi,
bilabial, boob, Caucasus, coco, Dada, deed, Emmett, Hannah,
horseshoer, intestines, Isis, mama, Mimi, murmur, noon, Otto, papa,
peep, reappear, redder, sees, Shanghaiings, Toto
\end{quote}

Write a function called {\tt IsDoubloon()} that returns {\tt TRUE}
if the given word is a doubloon and {\tt FALSE} otherwise.
\end{exercise}




%%%%%%%%%%%%%%%%%%%%%%%%%%%%%%%%%%

\begin{exercise}

The Captain Crunch decoder ring works by taking each letter in a
string and adding 13 to it.  For example, 'a' becomes 'n' and 'b'
becomes 'o'.  The letters ``wrap around'' at the end, so 'z' becomes
'm'.

\begin{enumerate}
\item  Write a function that takes a String and a character array to store the string 
containing the encoded version.  You should assume that the String
contains upper and lower case letters, and spaces, but no other
punctuation.  Lower case letters should be transformed into other lower
case letters; upper into upper.  You should not encode the spaces.

\item Generalize the Captain Crunch method so that instead of adding
13 to the letters, it adds any given amount.  Now you should be able
to encode things by adding 13 and decode them by adding -13.  Try it.

\end{enumerate}
\end{exercise}


%%%%%%%%%%%%%%%%%%%%%%%%%%%%%%%%%%

\begin{exercise}
In Scrabble each player has a set of tiles with letters on them, and
the object of the game is to use those letters to spell words.  The
scoring system is complicated, but as a rough guide longer words are
often worth more than shorter words.

Imagine you are given your set of tiles as a String, like {\tt
"qijibo"} and you are given another String to test, like {\tt "jib"}.
Write a function called {\tt TestWord()} that takes these two Strings and
returns true if the set of tiles can be used to spell the word.  You
might have more than one tile with the same letter, but you can only
use each tile once.
\end{exercise}



%%%%%%%%%%%%%%%%%%%%%%%%%%%%%%%%%%



\begin{exercise}
In real Scrabble, there are some blank tiles that can be used
as wild cards; that is, a blank tile can be used to represent
any letter.

Think of an algorithm for {\tt TestWord()} that deals with wild
cards.  Don't get bogged down in details of implementation like
how to represent wild cards.  Just describe the algorithm, using
English, pseudocode, or C.
\end{exercise}




