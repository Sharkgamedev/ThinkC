% LaTeX source for textbook ``How to think like a computer scientist''
% Copyright (C) 1999  Allen B. Downey
% Copyright (C) 2009  Thomas Scheffler


\selectlanguage{english}
\chapter{Strings and things}
\label{strings}

\section{Containers for strings}

We have seen four types of values---characters, integers,
floating-point numbers and strings---but only three types of
variables---{\tt char}, {\tt int} and {\tt double}.  So
far we have no way to store a string in a variable or perform
operations on strings.

This chapter is going to rectify this situation and I can now tell
you that strings in C are stored as an array of characters terminated by the 
character {\tt \textbackslash 0}.

By now this explanation should make sense to you and you probably understand
why we had to learn quite a bit about the working of the language 
before we could turn our attention towards string variables.

\index{<string.h>}
\index{header file!string.h}
In the previous chapter we have seen that operations on arrays have 
only minimal support from the C language itself and we had
to program extra functions by ourselves.
Fortunately things are a little bit easier when we manipulate
these special types of arrays - called strings.  There exist a number of
library functions in {\tt string.h} that make string handling a bit easier
than operations on pure arrays. 

Nevertheless string operations in 
C are still a lot more cumbersome than their equivalence in other
programing languages and can be a 
potential source of errors in your programs, if not handled 
carefully.

\section{String variables}

You can create a string variable as an array of characters in the following
way:

\begin{verbatim}
    char first[] = "Hello, ";
    char second[] = "world.";
\end{verbatim}
%
The first line creates an {\tt string} and assigns it the string value {\tt "Hello."}
In the second line we declare a second string variable. Remember,
the combined declaration and assignment is called initialization.

Initialisation time is the only time you can assign a value to a string directly (just as with
arrays in general). The initialisation parameters are passed 
in the form of a string constant enclosed in quotation marks ({\tt"}\ldots {\tt"}).

Notice the difference in syntax for the initialisation of arrays and strings. 
If you like you can also initialize the string in the normal array syntax, 
although this looks a little odd and is not very convenient to type.  

\begin{verbatim}
    char first[] = {'H','e','l','l','o',',',' ','\0'};
\end{verbatim}

There is no need to supply an array size when you are initialising the 
string variable at declaration time. The compiler compute the necessary
array size to store the supplied string.

Remember what we said about the nature of a string variable. It is an array of
characters \textbf{plus} a marker that shows where our string ends: the 
termination character  {\tt \textbackslash 0}.

Normally you do not have to supply this termination character.
The compiler understands our code and insertes it automatically.
However, in the example above, we treated our string exactly
like an array and in this case we have to insert the termination character ourselves.

When we are using a string variable to store different sting values 
during the lifetime of our program we have to declare a size big enough
for the largest sequence of characters that we are going to store.
We also have to make our string variable exactly one character longer than
the text we are going to store, because of the necessary termination character. 

We can output strings in the usual way using the {\tt printf()} function:

\begin{verbatim}
    printf("%s", first);
\end{verbatim}



%%
\section{Extracting characters from a string}

Strings are called ``strings'' because they are made up of a sequence,
or string, of characters.  The first operation we are going to
perform on a string is to extract one of the characters.  C
uses an index in square brackets ({\tt [} and {\tt ]}) for this operation:

\begin{verbatim}
    char fruit[] = "banana";
    char letter = fruit[1];
    printf ("%c\n", letter);
\end{verbatim}
%
The expression {\tt fruit[1]} indicates that I want character number 1
from the string named {\tt fruit}.  The result is stored in a {\tt
char} named {\tt letter}.  When I output the value of {\tt letter}, I
get a surprise:

\begin{verbatim}
   a
\end{verbatim}
%
{\tt a} is not the first letter of {\tt "banana"}.  Unless you are a
computer scientist.  For perverse reasons, computer scientists always
start counting from zero.  The 0th letter (``zeroeth'') of {\tt
"banana"} is {\tt b}.  The 1th letter (``oneth'') is {\tt a} and the
2th (``twoeth'') letter is {\tt n}.

If you want the zereoth letter of a string, you have to put
zero in the square brackets:

\begin{verbatim}
    char letter = fruit[0];
\end{verbatim}

\section{Length}
\index{string!length}
\index{length!string}
\index{<string.h>}
\index{header file!string.h}

To find the length of a string (the number of characters this string contains), we can
use the {\tt strlen()} function.  The function is called using the string variable
as an argument:

\begin{verbatim}
    #include <string.h>
    int main(void)
    { 
       int length;
       char fruit[] = "banana"; 

       length = strlen(fruit);
       return EXIT_SUCCESS;
    }   
\end{verbatim}
%
The return value of {\tt strlen()} in this case is 6. We assign this value to the integer
 {\tt length}  for further use.  

In order to compile this code, you need to include the
header file for the {\tt string.h} library. This library provides a number of
useful functions for operations on strings. 
%The type of the {\tt strlen()} is {\tt size_t}, an unsigned value large enough to
%enumerate any object that the system can handle (such as a string).
%for our example we can safely assume that the size of the string object 
%in our example will never
%exceed the range of the integer type. 
You should familiarize yourself with these functions because they can 
help you to solve your programming problems faster.


%Notice
%that it is legal to have a variable with the same name as a function.

To find the last letter of a string, you might be tempted to
try something like

\begin{verbatim}
    int length = strlen(fruit);
    char last = fruit[length];       /* WRONG!! */
\end{verbatim}
%
That won't work.  The reason is that  {\tt fruit} is still an array and there is no letter
at the array index  {\tt fruit[6]} in {\tt "banana"}.  Since we started counting at 0, the 6
letters are numbered from 0 to 5.  To get the last character,
you have to subtract 1 from {\tt length}.

\begin{verbatim}
    int length = strlen(fruit);
    char last = fruit[length-1];
\end{verbatim}

\section{Traversal}
\index{traverse}

A common thing to do with a string is
start at the beginning, select each character in turn, do
something to it, and continue until the end.  This pattern
of processing is called a {\bf traversal}.  A natural
way to encode a traversal is with a {\tt while} statement:

\begin{verbatim}
    int index = 0;
    while (index < strlen(fruit)) 
    {
        char letter = fruit[index];
        printf("%c\n" , letter);
        index = index + 1;
    }
\end{verbatim}
%
This loop traverses the string and outputs each letter on
a line by itself.  Notice that the condition is
{\tt index < strlen(fruit)}, which means that when
{\tt index} is equal to the length of the string, the
condition is false and the body of the loop is not executed.
The last character we access is the one with the
index {\tt strlen(fruit)-1}.

\index{loop variable}
\index{variable!loop}
\index{index}

The name of the loop variable is {\tt index}.  An {\bf
index} is a variable or value used to specify one member of an ordered
set, in this case the set of characters in the string.  The index
indicates (hence the name) which one you want.  The set has to be
ordered so that each letter has an index and each index
refers to a single character.

As an exercise, write a function that takes a {\tt string}
as an argument and that outputs the letters backwards, all on
one line.

%\section{A run-time error}
%\index{error!run-time}
%\index{run-time error}

%Way back in Section~\ref{run-time} I talked about run-time errors,
%which are errors that don't appear until a program has started
%running.

%So far, you probably haven't seen many run-time errors, because we
%haven't been doing many things that can cause one.  Well, now we are.
%If you use the {\tt []} operator and you provide an index that is
%negative or greater than {\tt length-1}, you will get a run-time
%error and a message something like this:

%\begin{verbatim}
%index out of range: 6, string: banana
%\end{verbatim}
%%
%Try it in your development environment and see how it looks.


%%
\section{Finding a  {\tt character} in a {\tt string}}
\label{Finding a  character in a string}

If we are looking for a letter in a {\tt string}, we have to search 
through the string and detect the position where this
letter occurs in the string.
Here is an implementation of this function:

\begin{verbatim}
    int LocateCharacter(char *s, char c)
    {
        int i = 0;
        while (i < strlen(s)) 
        {
            if (s[i] == c) return i;
            i = i + 1;
        }
        return -1;
    }
\end{verbatim}

We have to pass the {\tt string}
as the first argument, the other argument is the character
we are looking for. Our function returns the index of the first
occurrence of the letter, or {\tt -1} if the letter is not contained
in the string.


%%
\section{Pointers and Addresses}
\label{Pointers and Addresses}
\index{pointer}
\index{address}

When we look at the definition of the {\tt LocateCharacter()} function
you may notice the following construct {\tt char *s} which looks unfamiliar.

Remember, when we discussed how we had to pass
an array to a function, back in Section~\ref{Passing an array to a function},
we said that instead of copying the array, we only pass a reference to the function. 
Back then, we did not say exactly what this reference was.



C is one of the very few high-level programming languages that
let you directly manipulate objects in the computer memory.
In order to do this direct manipulation, we need to know the location 
of the object in memory: it's address. 
Adresses can be stored in variables of a special type.
These variables that point to other objects in memory 
(such as variables, arrays and strings) 
are therefore called {\bf pointer} variables. 

A pointer references the memory location of an object
and can be defined like this:

\begin{verbatim}
    int *i_p;
\end{verbatim}

This declaration looks similar to our earlier declarations, with one difference: the asterisk
in front of the name. 
We have given this pointer the type {\tt int}. The type specification has nothing to do
with the pointer itself, but rather defines which object this pointer is
supposed to reference (in this case an {\tt integer}).
This allows the compiler to do some type checking on, what would
otherwise be, an anonymous reference.
  
A pointer all by itself is rather meaningless, we also need an object that
this pointer is referencing:

\begin{verbatim}
    int number = 5; 
    int *i_p;
\end{verbatim}
 
This code-fragment defines an {\tt int} variable and a pointer. We can use 
 the "address-of" operator~{\tt \&} to assign  the memory 
 location or {\bf address} of our variable to the pointer.
 
 \begin{verbatim}
    i_p = &number;
\end{verbatim}

%{\tt}
Pointer {\tt i\_p} now references integer variable {\tt number}.
We can verify this using the "content-of" operator~{\tt *}.

 \begin{verbatim}
    printf("%i\n", *i_p);
\end{verbatim}

This prints {\tt 5}, which happens to be the content of the 
memory location at our pointer reference. 

With pointers we can directly manipulate memory locations:

 \begin{verbatim}
    *i_p = *i_p + 2;
    printf("%i\n", number);
\end{verbatim}

Our variable {\tt number} now has the value {\tt 7} and we begin to 
understand how our {\tt LocateCharacter()} function can directly access 
the values of string variables through the use of a {\tt char} pointer.

Pointers are widely used in many C programs and we have only
touched the surface of the topic. They can be immensely useful 
and efficient, however they can also be a potential source of
problems when not used appropriately. For this reason not
many programming languages support direct memory manipulation.
 

%If we are looking for a letter in an {\tt string}, we may
%not want to start at the beginning of the string.  One way
%to generalize the {\tt find} function is to write a version
%that takes an additional parameter---the index where we should
%start looking.  Here is an implementation of this function.

%\begin{verbatim}
%    int Find (char *s, char c, int i)
%    {
%        while (i < strlen(s)) 
%        {
%            if (s[i] == c) return i;
%            i = i + 1;
%        }
%        return -1;
%    }
%\end{verbatim}
%
%We have to pass the {\tt string}
%as the first argument.  The other arguments are the character
%we are looking for and the index where we should start.



%%
%\section{Looping and counting}
%\label{loopcount}
%\index{traverse!counting}
%\index{loop!counting}

%The following program counts the
%number of times the letter {\tt 'a'} appears in a string:

%\begin{verbatim}
%    char fruit[] = "banana";
%    int length = strlen(fruit);
%    int count = 0;

%    int index = 0;
%    while (index < length) 
%    {
%        if (fruit[index] == 'a') 
%        {
%            count ++;
%        }
%        index++;
%    }
%    printf ("%i\n", count);
%\end{verbatim}
%%
%This program demonstrates a common idiom, called a {\bf counter}.  The
%variable {\tt count} is initialized to zero and then incremented each
%time we find an {\tt 'a'}.  (To {\bf increment} is to increase by one;
%it is the opposite of {\bf decrement}.)  When we exit the loop, {\tt count}
%contains the result: the total number of a's.

%\index{counter}

%As an exercise, encapsulate this code in a function named
%{\tt CountLetters()}, and generalize it so that it accepts the
%string and the letter as arguments.
%% m�ssen wir die L�nge vorher ermitteln und �bergeben?

%\index{encapsulation}
%\index{generalization}

%As a second exercise, rewrite this function so that instead
%of traversing the string, it uses the version of
%{\tt find} we wrote in the previous section.


%\section{The {\tt strchr} function}
%\index{find}

%


% The {\tt strchr} function is like the opposite of the
%{\tt []} operator.  Instead of taking an index and extracting the
%character at that index, {\tt strchr} takes a character and finds the
%index where that character appears.

%\begin{verbatim}
%  char fruit[] = "banana";
%  int index = strchar(fruit,'a'));
%\end{verbatim}
%%
%This example finds the index of the letter {\tt 'a'} in the string.
%In this case, the letter appears three times, so it is not obvious
%what {\tt find} should do.  According to the documentation, it returns
%the index of the {\em first} appearance, so the result is 1.  If the
%given letter does not appear in the string, {\tt find} returns -1.

%In addition, there is a 
%version of {\tt find} that takes another {\tt string} as
%an argument and that finds the index where the substring
%appears in the string.  For example,

%\begin{verbatim}
%  apstring fruit = "banana";
%  int index = fruit.find("nan");
%\end{verbatim}
%%
%This example returns the value 2.




%%
%\pagebreak[4]

\section{String concatenation}

In Section~\ref{Finding a  character in a string} we have seen how we
could implement a search function that finds a {\tt character} in a {\tt string}.

One useful operation on strings is string {\bf concatenation}.  
To concatenate means to
join the two operands end to end.  For example:  {\tt shoe}
and {\tt maker} becomes {\tt shoemaker}.

Fortunately, we do not have to program all the necessary functions in C ourselves.
The {\tt string.h} library already provides several functions that we can
invoke on strings. 

We can use the library function {\tt strncat()} to concatenate
strings in C. 

\begin{verbatim}
    char fruit[20] = "banana";
    char bakedGood[] = " nut bread";
    strncat(fruit, bakedGood, 10);
    printf ("%s\n", fruit);
\end{verbatim}
%
The output of this program is {\tt banana nut bread}.

When we are using library functions it is important to completely understand
all the necessary arguments and to have a complete understanding
of the working of the function. 

The {\tt strncat()} does not take the two strings, joins them together and
produces a new combined string. It rather copies the content from the second
argument into the first. 

We therefore have to make sure that our first string is long enough to 
also hold the second string. We do this by defining the maximum capacity for 
string {\tt fruit} to be 19 characters + 1 termination character ({\tt char fruit[20]}). 
The third argument of {\tt strncat()}  specifies 
the number of characters
that will be copied from the second into the first string.



%It is also possible to concatenate a character onto the
%beginning or end of an {\tt string}.  In the following example, we
%will use concatenation and character arithmetic to output
%an abecedarian series.

%``Abecedarian'' refers to a series or list in which the elements
%appear in alphabetical order.  For example, in Robert McCloskey's book
%{\em Make Way for Ducklings}, the names of the ducklings are Jack,
%Kack, Lack, Mack, Nack, Ouack, Pack and Quack.  Here is a loop that
%outputs these names in order:

%\begin{verbatim}
%    char name[5];
%    char suffix[] = "ack";

%    char letter = 'J';
%    name[0] = letter;
%    name[1] = '\0';
%    
%    while (letter <= 'Q') 
%    {
%	/* Wrong, does not work, string gets longer and longer...*/        
%        printf("%s\n", strncat (name, suffix, 3));
%        letter++;
%        name[0] = letter;
%    }
%\end{verbatim}
%%
%The output of this program is:

%\begin{verbatim}
%Jack
%Kack
%Lack
%Mack
%Nack
%Oack
%Pack
%Qack
%\end{verbatim}
%%
%Of course, that's not quite right because I've misspelled ``Ouack''
%and ``Quack.''  As an exercise, modify the program to correct
%this error.

%Again, be careful to use string concatenation only with {\tt apstring}s
%and not with native C strings.  Unfortunately, an expression like
%{\tt letter + "ack"} is syntactically legal in C++, although it
%produces a very strange result, at least in my development environment.

%%
%\section{{\tt string}s are mutable}
%\index{immutable}
%\index{string}

%You can change the letters in an {\tt string} one at a time
%using the {\tt []} operator on the left side of an assignment.
%For example,

%\begin{verbatim}
%    char greeting[] = "Hello, world!";
%    greeting[0] = 'J';
%    printf ("%s", greeting);
%\end{verbatim}
%
%produces the output {\tt Jello, world!}.

\section{Assigning new values to {\tt string} variables}
\index{assigning!string}
\index{string}

So far we have seen how to initialise a string variable 
at declaration time. As with arrays in general, it is not 
legal to assign values directly to strings, because it is
not  possible to assign a value to an entire array.

\begin{verbatim}
    fruit = "orange";  /* Wrong: Cannot assign directly! */
\end{verbatim}

In order to assign a new value to an existing string variable we
have to use the {\tt strncpy()} function.
For example,

\begin{verbatim}
    char greeting[15];
    strncpy (greeting, "Hello, world!", 13);
\end{verbatim}

copies 13 characters from the of the second argument 
string to the first argument string.

This works, but not quite as expected. The {\tt strncpy()} function
copies exactly 13 characters from the second argument string
into the first argument string. And what happens to our 
string termination character  {\tt \textbackslash 0}?

%\pagebreak[4]

It is \textbf{not} copied automatically. We need to change
our copy statement to copy also the invisible 14th character at
the end of the string:

\begin{verbatim}
    strncpy (greeting, "Hello, world!", 14);
\end{verbatim}

However, if we only copy parts of the second string into the first
we need to explicitly set the n+1th character in the {\tt greeting[15]}
string to {\tt \textbackslash 0} afterwards.

\begin{verbatim}
    strncpy (greeting, "Hello, world!", 5); /*only Hello is copied*/
    greeting[5] = '\0';
\end{verbatim}

\vskip 1.5em

{\bf Attention!} In the last two sections we have used
the {\tt strncpy()} and the {\tt strncat()} function that require you
to explicitly supply the number of characters that will get copied
or attached to the first argument string. 

The {\tt string.h} library also defines the {\tt strcpy()} and 
the {\tt strcat()} functions that have no explicit bound on the number
of characters that are copied. 

The usage of these functions is
strongly discouraged! Their use has lead to a vast number
of security problems with C programs. Remember, C does not
check array boundaries and will continue copying characters
into computer memory even past the length of the variable.


%%
\section{{\tt string}s are not comparable}
\label{incomparable}
\index{comparison!string}
\index{string}

All the comparison operators that work on {\tt int}s and
{\tt double}s do work on {\tt strings}.  For example,
if you write the following code to determine if two strings are equal:

\begin{verbatim}
    if (word == "banana")  /* Wrong! */ 
\end{verbatim}

This test will always fail.

%
You have to use the {\tt strcmp()} function to compare two strings
with each other. The function returns {\tt 0} if the two strings are 
identical, a negative value if the first string is 'alphabetically less' than
the second (would be listed first in a dictionary) or a positive value
if the second string is 'greater'.

Please notice, this return value is not the standard true/false result, where the
return value {\tt 0}  is interpreted as 'false'.



The  {\tt strcmp()} function is useful for putting words
in alphabetical order.

\begin{verbatim}
    if (strcmp(word, "banana") < 0) 
    {
        printf( "Your word, %s, comes before banana.\n", word);
    } 
    else if (strcmp(word, "banana") > 0) 
    {
        printf( "Your word, %s, comes after banana.\n", word);
    } 
    else 
    {
        printf ("Yes, we have no bananas!\n");
    }
\end{verbatim}
%
You should be aware, though, that the {\tt strcmp()} function does
not handle upper and lower case letters the same way that people
do.  All the upper case letters come before all the lower case
letters.  As a result,

\begin{verbatim}
Your word, Zebra, comes before banana.
\end{verbatim}
%
A common way to address this problem is to convert strings to a
standard format, like all lower-case, before performing the
comparison.  The next sections explains how. 

%%
\section{Character classification}

\index{<ctype.h>}
\index{header file!ctype.h}
It is often useful to examine a character and test whether
it is upper or lower case, or whether it is a character or
a digit.  C provides a library of functions that perform
this kind of character classification.  In order to use these
functions, you have to include the header file {\tt ctype.h}.

\begin{verbatim}
    char letter = 'a';
    if (isalpha(letter)) 
    {
        printf("The character %c is a letter.", letter);
    }
\end{verbatim}
%
The return value from {\tt isalpha()} is an integer that is
0 if the argument is not a letter, and some non-zero value
if it is.

It is legal to use this kind of integer in a conditional, as shown
in the example.  The value {\tt 0} is treated as {\tt false}, and
all non-zero values are treated as {\tt true}.

%Technically, this sort of thing should not be allowed---integers are
%not the same thing as boolean values.  Nevertheless, the C habit of
%converting automatically between types can be useful.

Other character classification functions include {\tt isdigit()}, which
identifies the digits 0 through 9, and {\tt isspace()}, which identifies
all kinds of ``white'' space, including spaces, tabs, newlines, and a
few others.  There are also {\tt isupper()} and {\tt islower()}, which
distinguish upper and lower case letters.

Finally, there are two functions that convert letters from one
case to the other, called {\tt toupper()} and {\tt tolower()}.  Both take
a single character as an argument and return a (possibly
converted) character.

\begin{verbatim}
    char letter = 'a';
    letter = toupper (letter);
    printf("%c\n", letter);
\end{verbatim}
%
The output of this code is {\tt A}.

As an exercise, use the character classification and conversion
library to write functions named {\tt StringToUpper()} and
{\tt StringToLower()} that take a single string as
a parameter, and that modify the string by converting all the
letters to upper or lower case.  The return type should be
{\tt void}.

%%%
%\section{Other {\tt string} functions}

%This chapter does not cover all the {\tt apstring} functions.
%Two additional ones, {\tt c\_str} and {\tt substr}, are covered
%in Section~\ref{finput} and Section~\ref{parsing}.

\section{Getting user input}
\label{input}
\index{input!keyboard}

The programs we have written so far are pretty predictable;
they do the same thing every time they run.  Most of the time,
though, we want programs that take input from the user and
respond accordingly.

There are many ways to get input, including keyboard
input, mouse movements and button clicks, as well as more exotic
mechanisms like voice control and retinal scanning.  In this
text we will consider only keyboard input.

\index{scanf()}
\index{printf()}

In the header file {\tt stdio.h},
C defines a function named {\tt scanf()} that handles input in
much the same way that {\tt printf()} handles output.  We can use the following code to get an
integer value from the user:

\begin{verbatim}
    int x;
    scanf("%i", &x);
\end{verbatim}
%
The {\tt scanf()} function causes the program to stop executing and
wait for the user to type something.  If the user types a valid
integer, the program converts it into an integer value and
stores it in {\tt x}.

If the user types something other than an integer,
C doesn't report an error, or anything sensible like that.
Instead, the {\tt scanf()} function returns and leaves the value in {\tt x} unchanged.

Fortunately, there is a way to check and see if an input
statement succeeds.  The {\tt scanf()} function returns the number
of items that have been successfully read.
This number will be {\tt 1} when the last input
statement succeeded.  If not, we know that some previous operation
failed, and also that the next operation will fail.

Getting input from the user might look like this:

\begin{verbatim}
    int main (void)
    {
        int success, x;

        /* prompt the user for input */
        printf ("Enter an integer: \n");

        /* get input */
        success = scanf("%i", &x);

        /* check and see if the input statement succeeded */
        if (success == 1) 
        {
            /* print the value we got from the user */
            printf ("Your input: %i\n", x);
            return EXIT_SUCCESS;
        }
        printf("That was not an integer.\n");
        return EXIT_FAILURE;
    }
\end{verbatim}
%
There is another potential pitfall connected with the {\tt scanf()} function.
Your program code might want to insist that the user types a valid integer, because
this value is needed later on. In this case you might want to 
repeat the input statement in order to get a valid user input:
 
 \begin{verbatim}
    if (success != 1) 
    {
          while (success != 1)                                      
          { 
               printf("That was not a number. Please try again:\n");
               success = scanf("%i", &x);
          }  
     }
\end{verbatim}

\index{input!flushing the buffer}
\index{input buffer!flushing the buffer}
Unfortunately this code leads into an endless loop. You probably ask yourself, why?
The input from the keyboard is delivered to your program by the operating system, in 
something called an input buffer. A successful read operation automatically empties this buffer.
However, if the {\tt scanf()} function fails, like in our example, the buffer does not get emptied
and the next {\tt scanf()} operation re-reads the old value - you see the problem?

We need to empty the input buffer, before we can attempt to read the next input from the
user. Since there is no standard way to do this, we will introduce our own code that 
reads and empties the buffer using the {\tt getchar()} function. It run through a  {\tt while}-loop 
until there are no more characters left in the buffer (notice the construction of this loop, where all
the operations are executed in the test condition):
\begin{verbatim}
      char ch;    /* helper variable stores discarded chars*/
      while (success != 1)                                      
      { 
          printf("That isn't a number. Please try again:\n");
           /* now we empty the input buffer*/
           while ((ch = getchar()) != '\n' && ch != EOF);
           success = scanf("%i", &x);
      }    

\end{verbatim}
 

The {\tt scanf()} function can also be used to input a {\tt string}:

\begin{verbatim}
    char name[80];

    printf ("What is your name?");
    scanf ("%s", name);
    printf ("%s", name);
\end{verbatim}
%
Again, we have to make sure our string variable is large enough
to contain the complete user input. Notice the difference in the
argument of the {\tt scanf()} function when we are reading 
an {\tt integer} or a {\tt string}. The function requires a
pointer to the variable where the input value will be stored.
If we are reading an {\tt integer} we need to use the address operator {\tt \&}
with the variable name. In the case of a {\tt string} we simply provide the 
variable name.

Also notice, that the {\tt scanf()} function only takes the first word of
the input, and leaves the rest for the next input statement.
So, if you run this program and type your full name, it
will only output your first name.



\section{Glossary}

\begin{description}

\item[index:]  A variable or value used to select one of the
members of an ordered set, like a character from a string.

\item[traverse:]  To iterate through all the elements of a set
performing a similar operation on each.

\item[counter:]  A variable used to count something, usually
initialized to zero and then incremented.

\item[concatenate:] To join two operands end-to-end.

\item[pointer:] A reference to an object in computer memory.

\item[address:] The exact storage location of objects in memory.

\index{index}
\index{traverse}
\index{counter}
\index{increment}
\index{decrement}
\index{concatenate}
\index{pointer}
\index{address}


\end{description}

\section{Exercises}
\setcounter{exercisenum}{0}


% LaTeX source for textbook ``How to think like a computer scientist''
% Copyright (C) 1999  Allen B. Downey
% Copyright (C) 2009  Thomas Scheffler

%%%%%%%%%%%%%%%%%%%%%%%%%%%%%%%%%%%%%%
\begin{exercise}
A word is said to be ``abecedarian'' if the letters in the
word appear in alphabetical order.  For example, the following
are all 6-letter English abecedarian words.

\begin {quote}
abdest, acknow, acorsy, adempt, adipsy, agnosy, befist, behint,
beknow, bijoux, biopsy, cestuy, chintz, deflux, dehors, dehort,
deinos, diluvy, dimpsy
\end{quote}

\begin{enumerate}

\item Describe an algorithm for checking whether a given word (String)
is abecedarian, assuming that the word contains only lower-case
letters.  Your algorithm can be iterative or recursive.

\item Implement your algorithm in a function called {\tt IsAbecedarian()}.

\end{enumerate}
\end{exercise}

%%%%%%%%%%%%%%%%%%%%%%%%%%%%%%%%%%%%%%


\begin{exercise}
Write a function called {\tt LetterHist()} that takes a string and and an int array as
parameters. The function builds a histogram by filling the count of each letter in the string to the array.
The zeroeth element of the histogram array should contain the number of a's
in the String (upper and lower case); the 25th element should contain
the number of z's.
Your solution should only traverse the String once.
\end{exercise}


%%%%%%%%%%%%%%%%%%%%%%%%%%%%%%%%%%%%%%


\begin{exercise}
A word is said to be a ``doubloon'' if every letter that appears in the
word appears exactly twice.  For example, the following are all the
doubloons I found in my dictionary.

\begin {quote}
Abba, Anna, appall, appearer, appeases, arraigning, beriberi,
bilabial, boob, Caucasus, coco, Dada, deed, Emmett, Hannah,
horseshoer, intestines, Isis, mama, Mimi, murmur, noon, Otto, papa,
peep, reappear, redder, sees, Shanghaiings, Toto
\end{quote}

Write a function called {\tt IsDoubloon()} that returns {\tt TRUE}
if the given word is a doubloon and {\tt FALSE} otherwise.
\end{exercise}




%%%%%%%%%%%%%%%%%%%%%%%%%%%%%%%%%%

\begin{exercise}

The Captain Crunch decoder ring works by taking each letter in a
string and adding 13 to it.  For example, 'a' becomes 'n' and 'b'
becomes 'o'.  The letters ``wrap around'' at the end, so 'z' becomes
'm'.

\begin{enumerate}
\item  Write a function that takes a String and that returns a new String
containing the encoded version.  You should assume that the String
contains upper and lower case letters, and spaces, but no other
punctuation.  Lower case letters should be transformed into other lower
case letters; upper into upper.  You should not encode the spaces.

\item Generalize the Captain Crunch method so that instead of adding
13 to the letters, it adds any given amount.  Now you should be able
to encode things by adding 13 and decode them by adding -13.  Try it.

\end{enumerate}
\end{exercise}


%%%%%%%%%%%%%%%%%%%%%%%%%%%%%%%%%%

\begin{exercise}
In Scrabble each player has a set of tiles with letters on them, and
the object of the game is to use those letters to spell words.  The
scoring system is complicated, but as a rough guide longer words are
often worth more than shorter words.

Imagine you are given your set of tiles as a String, like {\tt
"qijibo"} and you are given another String to test, like {\tt "jib"}.
Write a function called {\tt TestWord()} that takes these two Strings and
returns true if the set of tiles can be used to spell the word.  You
might have more than one tile with the same letter, but you can only
use each tile once.
\end{exercise}



%%%%%%%%%%%%%%%%%%%%%%%%%%%%%%%%%%



\begin{exercise}
In real Scrabble, there are some blank tiles that can be used
as wild cards; that is, a blank tile can be used to represent
any letter.

Think of an algorithm for {\tt TestWord()} that deals with wild
cards.  Don't get bogged down in details of implementation like
how to represent wild cards.  Just describe the algorithm, using
English, pseudocode, or C.
\end{exercise}






